\documentclass[12pt]{article}

\begin{document}
\def\GeV{\mbox{GeV}}
\title{MCFM}
\section{Input parameters}
The default input parameters are given in the table above.
\begin{table}
\begin{center}
\begin{tabular}{|c|c|c|c|} \hline
Parameter & Fortran name & Default value & Location where set                 \cr 
\hline
$m_t$            & {\tt mt}      & 175 GeV             &{\tt /User/mdata.f}   \cr
$m_\tau$         & {\tt mtau}      & 1.777 GeV           &{\tt /User/mdata.f}   \cr
$m^2_\tau(Q=100\GeV)$& {\tt mtausq}  & 3.1602 GeV$^2$      &{\tt /User/mdata.f}   \cr
$m^2_c(Q=100\GeV)$   & {\tt mcsq}    & 0.4  GeV$^2$        &{\tt /User/mdata.f}   \cr
$m^2_b(Q=100\GeV)$   & {\tt mbsq}    & 10.7 GeV$^2$        &{\tt /User/mdata.f}   \cr
$\Gamma_\tau$    & {\tt tauwidth}& 2.269$\times$10$^{-12}$~GeV
                                                       &{\tt /User/mdata.f}   \cr
$M_Z$            & {\tt zmass}   & 91.187 GeV          &{\tt /User/mdata.f}   \cr
$\Gamma_Z$       & {\tt zwidth}  & 2.49 GeV            &{\tt /User/mdata.f}   \cr
$M_W$            & {\tt wmass}   & 81.41 GeV           &{\tt /User/mdata.f}   \cr
$\Gamma_W$       & {\tt wwidth}  & 2.06 GeV            &{\tt /User/mdata.f}   \cr
$\alpha(M_Z)$    & {\tt aemmz}   & 1/128.89            &{\tt /User/mdata.f}   \cr
$G_F$            & {\tt gf}      & 1.16639$\times$10$^{-5}$
                                                       &{\tt /Inc/constants.f}\cr
$\sin^2 \theta_w$& {\tt xw}      & Calculated          & See text             \cr
$g^2_w$          & {\tt gwsq}    & Calculated          & See text             \cr
$V_{ud}$         & {\tt Vud}     & 0.975               &{\tt /User/mdata.f}   \cr
$V_{us}$         & {\tt Vus}     & 0.22220486          &{\tt /User/mdata.f}   \cr
$V_{ub}$         & {\tt Vub}     & 0.                  &{\tt /User/mdata.f}   \cr
$V_{cd}$         & {\tt Vcd}     & 0.22220486          &{\tt /User/mdata.f}   \cr
$V_{cs}$         & {\tt Vcs}     & 0.975               &{\tt /User/mdata.f}   \cr
$V_{cb}$         & {\tt Vcb}     & 0.                  &{\tt /User/mdata.f}   \cr
\hline
\end{tabular}
\label{default} 
\end{center}
\end{table}
We use the effective field theory approach, which is good for all
except couplings to $b$-quarks (this may be added later). This is
valid for scales below the top mass.
      $$e^2 =  4 \pi \alpha(M_Z)$$
      $$g_w^2 =  8 M^2_W  \frac{G_F}{\sqrt{2}} $$
      $$\sin \theta_w =  \frac{e^2}{g_w^2}$$
For further details, see Georgi, {\it Nucl. Phys.} {\bf B363} (1991)
301. 

\subsection{Parton distributions}
The value of $\alpha_S(M_Z)$ is not adjustable; it is hardwired with the
parton distribution. A collection of modern parton  distribution functions
are included. 
The distributions, together with their associated $\alpha_S(M_Z)$ values, 
are given in the table below. For the older distributions, where the coupling 
was specified by $\Lambda$ this requires some calculation/guesswork.
\begin{table}[h]
\begin{center}
\begin{tabular}{|c|c||c|c|}
\hline
{\tt mrs99\_1}  & 0.1175    & {\tt mrs99\_2}  & 0.1175      \cr
{\tt mrs99\_3}  & 0.1175    & {\tt mrs99\_4}  & 0.1125      \cr    
{\tt mrs99\_5}  & 0.1225    & {\tt mrs99\_6}  & 0.1178      \cr    
{\tt mrs99\_7}  & 0.1171    & {\tt mrs99\_8}  & 0.1175      \cr    
{\tt mrs99\_9}  & 0.1175    & {\tt mrs9910}  & 0.1175      \cr    
{\tt mrs9911}  & 0.1175    & {\tt mrs9912}  & 0.1175      \cr    
{\tt mrs98z1}  &  0.1175      & {\tt cteq5hq}  &  0.118  \cr
{\tt mrs98z2}  &  0.1175      & {\tt cteq5f3}  &  0.106  \cr
{\tt mrs98z3}  &  0.1175      & {\tt cteq5f4}  &  0.112  \cr
{\tt mrs98z4}  &  0.1125      & {\tt mtungb1}  &  0.109  \cr
{\tt mrs98z5}  &  0.1225      & {\tt cteq4\_m} &  0.116  \cr
{\tt mrs96r1}  &  0.113       & {\tt cteq4\_d} &  0.116  \cr
{\tt mrs96r2}  &  0.120       & {\tt cteq4\_l} &  0.132  \cr
{\tt mrs96r3}  &  0.113       & {\tt cteq4a1}  &  0.110  \cr
{\tt mrs96r4}  &  0.120       & {\tt cteq4a2}  &  0.113  \cr
{\tt hmrs90e}  &  0.098382675 & {\tt cteq4a3}  &  0.116  \cr
{\tt hmrs90b}  &  0.107961191 & {\tt cteq4a4}  &  0.119  \cr
{\tt cteq5\_m} &  0.118       & {\tt cteq4a5}  &  0.122  \cr
{\tt cteq5\_d} &  0.118       & {\tt cteq4hj}  &  0.116  \cr
{\tt cteq5\_l} &  0.127       & {\tt cteq4lq}  &  0.114  \cr
{\tt cteq5hj}  &  0.118       &   &    \cr
\hline
\end{tabular}
\end{center}
\label{pdlabel}
\end{table}

Alternatively it would be simple to write a link to PDFLIB.  

\section{Runtime options}
Run time options are chosen by modifying the file 
{\tt /bin/options.DAT}. Each line of this file is of the form
\begin{displaymath}
{\tt value} \hspace{3cm} [{\tt parameter}]
\end{displaymath}
and we will give a description of all the parameters below, together with
valid and/or sensible inputs for ${\tt value}$.

\begin{itemize}
\item {\tt nproc}.
The process to be studied is given by
choosing a process number, according to the following table.
$f(p_i)$ denotes a generic partonic jet.
\begin{table}
\begin{center}
\begin{tabular}{|l|l|}
\hline
{\tt nproc} & $f(p_1)+f(p_2) \to \ldots $ \cr
\hline
0   & $ W^+ ~\mbox{for total cross-section--No $W$-BR} $ \cr 
1   & $ W^+(\to \nu(p_3)+e^+(p_4)) $ \cr 
5   & $ W^- ~\mbox{for total cross-section--No $W$-BR} $ \cr 
6   & $ W^-(\to e^-(p_3)+\bar{\nu}(p_4)) $ \cr 
\hline
11  & $ f(p_5)+W^+(\to \nu(p_3)+e^+(p_4)) $ \cr 
16  & $ f(p_5)+W^-(\to e^-(p_3)+\bar{\nu}(p_4)) $ \cr 
\hline
21  & $ b(p_5)+\bar{b}(p_6) +W^+(\to \nu(p_3)+e^+(p_4)) $ \cr 
22  & $ f(p_5)+f(p_6) +W^+(\to \nu(p_3)+e^+(p_4)) $ \cr 
26  & $ b(p_5)+\bar{b}(p_6) +W^-(\to e^-(p_3)+\bar{\nu}(p_4)) $ \cr 
27  & $ f(p_5)+f(p_6) +W^-(\to e^-(p_3)+\bar{\nu}(p_4)) $ \cr 
\hline
30  & $ Z^0 ~\mbox{for total cross-section--No $Z$-BR} $ \cr 
31  & $ Z^0(\to e^-(p_3)+e^+(p_4)) $ \cr 
32  & $ Z_0(\to 3*(\nu(p_3)+\bar{\nu}(p_4))) $ \cr 
33  & $ Z^0(\to b(p_3)+\bar{b}(p_4)) $ \cr 
\hline
41  & $ f(p_5)+Z^0(\to e^-(p_3)+e^+(p_4)) $ \cr 
42  & $ f(p_5)+Z_0(\to 3*(\nu(p_3)+\bar{\nu}(p_4)))-(\mbox{sum over 3} \nu) $ \cr 
43  & $ f(p_5)+Z^0(\to b(p_3)+\bar{b}(p_4)) $ \cr 
44  & $ f(p_5)+f(p_6)+Z^0(\to e^-(p_3)+e^+(p_4)) $ \cr 
\hline
51  & $ b(p_5)+\bar{b}(p_6)+Z^0(\to e^-(p_3)+e^+(p_4)) $ \cr 
52  & $ b(p_5)+\bar{b}(p_6)+Z_0(\to 3*(\nu(p_3)+\bar{\nu}(p_4))) $ \cr 
53  & $ b(p_5)+\bar{b}(p_6)+Z^0(\to b(p_3)+\bar{b}(p_4)) $ \cr 
\hline
60  & $ W^+ + W^- ~\mbox{for total cross-section--No $W$-BRs} $ \cr 
61  & $ W^+(\to \nu(p_3)+e^+(p_4)) +W^-(\to e^-(p_5)+\bar{\nu}(p_6)) $ \cr 
62  & $ W^+(\to \nu(p_3)+e^+(p_4)) +W^-(\to q(p_5)+{\bar q}(p_6)) $ \cr 
63  & $ W^+(\to q(p_3)+{\bar q}(p_4)) +W^-(\to e^-(p_5)+\bar{\nu}(p_6)) $ \cr 
\hline
70  & $ Z^0 + W^+ ~\mbox{for total cross-section--No $W$or $Z$-BRs} $ \cr 
71  & $ Z^0(\to e^-(p_5)+e^+(p_6))+W^+(\to \nu(p_3)+\mu^+(p_4)) $ \cr 
72  & $ Z^0(\to 3*(\nu(p_5)+\bar{\nu}(p_6)))+W^+(\to \nu(p_3)+e^+(p_4)) $ \cr 
73  & $ Z^0(\to b(p_5)+\bar{b}(p_6))+W^+(\to \nu(p_3)+e^+(p_4)) $ \cr 
\hline
75  & $ Z^0 + W^- ~\mbox{for total cross-section--No $W$or $Z$-BRs} $ \cr 
76  & $ Z^0(\to e^-(p_5)+e^+(p_6))+W^-(\to \mu^-(p_3)+\bar{\nu}(p_4)) $ \cr 
77  & $ Z^0(\to \nu(p_5)+\bar{\nu}(p_6))+W^-(\to e^-(p_3)+\bar{\nu}(p_4)) $ \cr 
78  & $ Z^0(\to b(p_5)+\bar{b}(p_6))+W^-(\to e^-(p_3)+\bar{\nu}(p_4)) $ \cr 
\hline
\end{tabular}
\end{center}
\end{table}

\begin{table}
\begin{center}
\begin{tabular}{|l|l|}
\hline
{\tt nproc} & $f(p_1)+f(p_2) \to \ldots $ \cr
\hline
80  & $ Z^0 + Z^0 ~\mbox{for total cross-section--No $Z$-BRs} $ \cr 
81  & $ Z^0(\to e^-(p_5)+e^+(p_6))+Z^0(\to \mu^-(p_3)+\mu^+(p_4)) $ \cr 
82  & $ Z^0(\to e^-(p_5)+e^+(p_6))+Z^0(\to 3*(\nu(p_3)+\bar{\nu}(p_4))) $ \cr 
83  & $ Z^0(\to e^-(p_5)+e^+(p_6))+Z^0(\to b(p_3)+\bar{b}(p_4)) $ \cr 
84  & $ Z^0(\to b(p_5)+\bar{b}(p_6))+Z^0(\to 3*(\nu(p_3)+\bar{\nu}(p_4))) $ \cr 
\hline
85  & $ Z^0 + Z^0 ~\mbox{for total cross-section} ~(\mathrm{no}~\gamma^*) $ \cr 
86  & $ Z^0(\to e^-(p_5)+e^+(p_6))+Z^0(\to \mu^-(p_3)+\mu^+(p_4)) ~(\mathrm{no}~\gamma^*) $ \cr 
87  & $ Z^0(\to e^-(p_5)+e^+(p_6))+Z^0(\to 3*(\nu(p_3)+\bar{\nu}(p_4))) ~(\mathrm{no}~\gamma^*) $ \cr 
88  & $ Z^0(\to e^-(p_5)+e^+(p_6))+Z^0(\to b(p_3)+\bar{b}(p_4)) ~(\mathrm{no}~\gamma^*) $ \cr 
89  & $ Z^0(\to
b(p_5)+\bar{b}(p_6))+Z^0(\to 3*(\nu(p_3)+\bar{\nu}(p_4))) ~(\mathrm{no}~\gamma^*) $ \cr 
\hline
90  & $ H + W^+ ~\mbox{for total cross-section--No $W$or $H$-BRs} $ \cr 
91  & $ H(\to b(p_5)+\bar{b}(p_6)) + W^+(\to \nu(p_3)+e^+(p_4)) $ \cr 
96  & $ H(\to b(p_5)+\bar{b}(p_6)) + W^-(\to e^-(p_3)+\bar{\nu}(p_4)) $ \cr 
\hline
100 & $ H + Z^0 ~\mbox{for total cross-section--No $Z$or $H$-BRs} $ \cr 
101 & $ H(\to b(p_5)+\bar{b}(p_6)) + Z^0(\to e^-(p_3)+e^+(p_4)) $ \cr 
102 & $ H(\to b(p_5)+\bar{b}(p_6)) + Z^0(\to 3*(\nu(p_3)+\bar{\nu}(p_4))) $ \cr 
103 & $ H(\to b(p_5)+\bar{b}(p_6)) + Z^0(\to b(p_3)+\bar{b}(p_4)) $ \cr 
\hline
110 & $ H(\to W^- W^+) ~\mbox{for total cross-section--No $H$or $W$-BRs} $ \cr 
111 & $ H(\to W^-(e^-(p_5)+\bar{\nu}(p_6))+W^+(\nu(p_3)+e^+(p_4))) $ \cr 
\hline
121 & $ H(\to Z^0(e^-(p_5)+e^+(p_6))+Z^0(\mu^-(p_3)+\mu^+(p_4)) $ \cr 
122 & $ H(\to Z^0(e^-(p_5)+e^+(p_6))+Z^0(3*(\nu(p_3)+\bar{\nu}(p_4))) $ \cr 
\hline
150 & $ t \bar{t} ~\mbox{for total cross-section} $ \cr 
151 & $ t(\to\nu(p_3)+e^+(p_4)+b(p_5))+\bar{b}(p_6))+W^-(\to e^-(p_7)+\bar{\nu}(p_8))~(\mathrm{missing}) $ \cr 
152 & $ t(\to\nu(p_3)+e^+(p_4)+b(p_5))+\bar{b}(p_6))+W^-(\to q(p_7)+{\bar q}(p_8))~(\mathrm{missing}) $ \cr 
\hline
161 & $ t(\to\nu(p_3)+e^+(p_4)+b(p_5))+\bar{b}(p_6))+q(p_7)~(\mathrm{missing}) $ \cr 
\hline
171 & $ t(\to \nu(p_3)+e^+(p_4)+b(p_5))+\bar{b}(p_6)) $ \cr 
\hline
180 & $ \tau^- \tau^+ ~\mbox{for total cross-section-Drell-Yan} $ \cr 
181 & $ \tau^-(\to \nu(p_3)+e^+(p_4)+\nu(p_5))+\tau^+(\to \bar{\nu}(p_6)+e^-(p_8)+\bar{\nu}(p_9)) $ \cr  
\hline
190 & $ t(p_3) + {\bar t}(p_4) + H(p_5) $ \cr 
191 & $ t(p_3+p_4+p_5) + {\bar t}(p_6+p_7+p_8) + H(p_9+p_{10}) $ \cr 
192 & $ t(\to \nu(p_3)+e^+(p_4)+b(p_5)) +
{\bar t}(\to {\bar b}(p_6)+{\bar \nu}(p_7)+e^-(p_8)) + H(p_9+p_{10}) $ \cr 
\hline
200 & $ H(p_3+p_4) + f(p_5) ~\mbox{for total cross-section-No $H$ BR} $ \cr 
201 & $ H(\to \tau(p_3)+{\bar \tau}(p_4)) + f(p_5)$ \cr 
202 & $ H(\to \tau^- (\to \nu_\tau(p_3)+e^-(p_4)+{\bar \nu_e}(p_5))
+\tau^+(\to {\bar \nu_\tau}(p_6)+\nu_e(p_7)+e^+(p_8)) + f(p_9)$ \cr 
\hline
30$n$ &  Check of volume of $n$-particle phase space \cr
\hline
\end{tabular}
\end{center}
\end{table}

\item {\tt part}.
This parameter has 4 possible values, described below:
\begin{itemize}
\item {\tt lord}.
The calculation is performed at leading order only.
\item {\tt virt}.
Virtual (loop) contributions to the next-to-leading order result are
calculated, (+counterterms to make them finite).
\item {\tt real}.
In addition to the loop diagrams calculated by {\tt virt}, the full
next-to-leading order results must include contributions from diagrams
involving real gluon emission (-counterterms to make them finite).
\item {\tt tota}.
For simplicity, the {\tt tota} option simply runs the {\tt virt} and
{\tt real} real pieces in series before performing a sum to obtain
the full next-to-leading order result. In practise, it may be more
efficient to do this by hand as described below.
\end{itemize}

\item {\tt runstring}.
When {\tt MCFM} is run, it will write output to several files. The
label {\tt runstring} (up to 9 characters long) will be appended
to the names of these files.

\item {\tt verbose}.
Setting this flag to {\tt .false.} will disable certain output in
the initialization stage of the program. It is recommended that
this be left at the default value {\tt .true.}.

\item {\tt sqrts}. This is the centre-of-mass energy, $\sqrt{S}$ of
the colliding particles, measured in GeV.

\item {\tt ih1}, {\tt ih2}. The identities of the incoming hadrons
may be set with these parameters, allowing simulations for both
$p{\bar p}$ (such as the Tevatron) and $pp$ (such as the LHC)
simulations. Setting {\tt ih1} equal to ${\tt +1}$ corresponds to
a proton, whilst ${\tt -1}$ corresponds to an anti-proton.

\item {\tt pdlabel}. The choice of parton distribution is made by
inserting the appropriate 7-character code from Table~{\ref{pdlabel}}
here. As mentioned above, this also sets the value of $\alpha_S(M_Z)$.

\item {\tt hmass}. For processes involving the Higgs boson, this
parameter should be set equal to the putative value of $M_H$.

\item {\tt scale}. This parameter may be used to adjust the value
of the renormalization and factorization scale. This is the scale
at which $\alpha_S$ is evaluated and will typically be set to either
a mass scale appropriate to the process ($M_W$, $M_Z$, $M_t$ for
instance).

\item {\tt m34min}, {\tt m34max}, {\tt m56min}, {\tt m56max}.
These parameters represent a basic set of cuts that may be applied
to the calculated cross-section. The only events that contribute to
the cross-section will have, for example,
{\tt m34min} $<$ {\tt m34} $<$ {\tt m34max} where {\tt m34} is the
invariant mass of particles 3 and 4 that are specified by {\tt nproc}.

\item {\tt rtsmin}. This provides a cut-off on the lowest generated
$\hat{s}$ for the partonic event. This cut-off is required for
numerical accuracy and should not be altered.

\item {\tt zerowidth}. When set to {\tt .true.} then all vector
bosons are produced on-shell. This is appropriate for calculations
of {\it total} cross-sections such as {\tt nproc} $=$ {\tt 60},
{\tt 70} and {\tt 80}. When interested in decay products of the
bosons (such as {\tt nproc} $=$ {\tt 61}), this should be set
to {\tt .false.}.

\item {\tt makecuts}. If this parameter is set to {\tt .false.} then
no additional cuts are applied to the events. Otherwise, events will
be rejected according to a set of cuts that is specified by routines
in {\tt /src/User/} and which varies according to the process
specified in {\tt nproc}. 

\item {\tt Rcut}. If the final state of the chosen process contains
either quarks or gluons then for each event an attempt will be made
to form them into jets. For this it is necessary to define the
jet separation $\Delta R=\sqrt{{\Delta \eta}^2 + {\Delta \phi}^2}$
so that after jet combination, all jet pairs are separated by
$\Delta R >$~{\tt Rcut}. All jets will also satisfy the hard-coded
cuts $E_T^{\rm jet} > 15$~GeV, $|\eta^{\rm jet}| < 2.5$.

\item {\tt itmx1}, {\tt itmx2}. The program will perform two runs of
{\tt VEGAS} - once for prec-conditioning and then the final run to
collect the total cross-section and fill histograms. The number of
sweeps for each run is given by {\tt itmx1} (pre-conditioning)
and {\tt itmx2} (final). The recommended values are the default,
{\tt 10}.

\item {\tt ncall1}, {\tt ncall2}. For every sweep of {\tt VEGAS},
the number of events generated will be {\tt ncall1} in the
pre-conditioning stage and {\tt ncall2} in the final run. The number
of events required depends upon a number of factors. The error
estimate on a total cross-section will often be reasonable for a
fairly small number of events, whereas accurate histograms will
require a longer run. As the number of particles in the final state
increases, so should the number of calls per sweep. Typically one
might make trial runs with {\tt part} set to {\tt lord} to determine
reasonable values for {\tt ncall1} and {\tt ncall2}. Such values
should also be appropriate for the {\tt virt} piece of
next-to-leading order and should probably be increased by a factor of
$\sim 5$ for the {\tt real} calculation.

\item {\tt ij}. This is the seed for the {\tt VEGAS} integration
and can be altered to give different results for otherwise identical
runs.

\item {\tt realwt}. This is a technical parameter and its value
should be left unchanged ({\tt .true.}).

\item {\tt cutoff}. This is a technical parameter and its value
should be left unchanged ({\tt 10}).

\item {\tt dryrun}. The default value of this parameter is
{\tt .false.}. When set to {\tt .true.} the pre-conditioning sweeps
in the {\tt VEGAS} integration are skipped, with the reported
results coming from a single run.

\item {\tt debug}. This is useful as a debugging measure and as
such it should be left unchanged ({\tt .false.}).

\end{itemize}

\section{Output}
 
In addition to the direct output of the program to {\tt stdout}, after
the final sweep of {\tt VEGAS} the program will output two additional files.

The standard output will detail the iteration-by-iteration best estimate
of the total cross-section, together with the accompanying error estimate.
After all sweeps have been completed, a final summary line will be printed.
In the {\tt npart}~$=$~{\tt tota} case, this last line will actually be the
sum of the two separate real and virtual integrations.

The two other files that are output will be {\tt outputname.dat} and
{\tt outputname.top}, which contain data for various histograms associated
with the calculated process. The first of these is in a raw format 
which may be read in by a plotting package of the user's choosing. The
other file contains the histograms as a {\tt TOPDRAWER} file. The structure
of {\tt outputname} is as follows:
\begin{displaymath}
{\tt procname\_part\_pdlabel\_scale\_runstring}
\end{displaymath}
where {\tt procname} is a label assigned by the program corresponding to
the calculated process; the remaining labels are as input by the user
in the file {\tt options.DAT}.
 
\end{document}

