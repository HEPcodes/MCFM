\documentclass[12pt]{article}

\begin{document}
\def\GeV{\mbox{GeV}}
\def\cteqsixsixm{\mbox{0802.0007 [hep-ph]}}
\def\cteqsixonem{\mbox{hep-ph/0303013}}
\def\cteqsix{\mbox{hep-ph/0201195}}
\def\cteqfive{\mbox{hep-ph/9903282}}
\def\cteqfour{\mbox{hep-ph/9606399}}
\def\cteqthree{\mbox{MSU-HEP/41024}}
\def\mrstff{\mbox{hep-ph/0603143}}
\def\mrstohtwo{\mbox{hep-ph/0211080}}
\def\mrstohone{\mbox{hep-ph/0110215}}
\def\mrsninenine{\mbox{hep-ph/9907231}}
\def\mrsnineeight{\mbox{hep-ph/9803445}}
\def\mrsninesix{\mbox{PLB387 (1996) 419}}
\def\mrsninefive{\mbox{PLB354 (1995) 155}}
\def\hmrs{\mbox{Durham DTP-90-04}}
\def\mstwoheight{\mbox{0901.0002 [hep-ph]}}

\thispagestyle{empty}
\vspace*{3cm}
\begin{center}
{\Huge MCFM v5.6} \\
\vspace*{0.5cm}
\Large{A Monte Carlo for FeMtobarn} \\
\Large{processes at Hadron Colliders} \\
\vspace*{2cm}
{\huge Users Guide} \\
\vspace*{4cm}
{\it Authors:} \\
\vspace*{0.2cm}
J. M. Campbell ({\tt J.Campbell@physics.gla.ac.uk}) \\
R. K. Ellis ({\tt ellis@fnal.gov}) \\
\vspace*{2.5cm}
{\it \small Updated: 11th August 2009}
\end{center}

\newpage

\tableofcontents

\section{Overview}

MCFM is a parton-level Monte Carlo program which gives NLO predictions
for a range of processes at hadron colliders. The preferred reference
to the program depends upon the process under study. The relevant papers
are:
\begin{itemize}
\item J.~M.~Campbell and R.~K.~Ellis, \\
  {\it ``An update on vector boson pair production at hadron colliders,''} \\
  Phys.\ Rev.\ D {\bf 60}, 113006 (1999)
  [arXiv:hep-ph/9905386].
\item J.~M.~Campbell and R.~K.~Ellis, \\
  {\it ``Radiative corrections to Z b anti-b production,''} \\
  Phys.\ Rev.\ D {\bf 62}, 114012 (2000)
  [arXiv:hep-ph/0006304].
\item J.~Campbell and R.~K.~Ellis, \\
  {\it ``Next-to-leading order corrections to W + 2jet and Z + 2jet production  at
   hadron colliders,''} \\
  Phys.\ Rev.\ D {\bf 65}, 113007 (2002)
  [arXiv:hep-ph/0202176].
\item J.~Campbell, R.~K.~Ellis, F.~Maltoni and S.~Willenbrock, \\
  {\it ``Higgs boson production in association with a single bottom quark,''} \\
  Phys.\ Rev.\ D {\bf 67}, 095002 (2003)
  [arXiv:hep-ph/0204093].
\item J.~Campbell, R.~K.~Ellis and D.~L.~Rainwater, \\
  {\it ``Next-to-leading order QCD predictions for W + 2jet and Z + 2jet  production
     at the CERN LHC,''} \\
  Phys.\ Rev.\ D {\bf 68}, 094021 (2003)
  [arXiv:hep-ph/0308195].
\item J.~Campbell, R.~K.~Ellis, F.~Maltoni and S.~Willenbrock, \\
  {\it ``Associated production of a Z boson and a single heavy-quark jet,''} \\
  Phys.\ Rev.\ D {\bf 69}, 074021 (2004)
  [arXiv:hep-ph/0312024].
\item E.~L.~Berger and J.~Campbell, \\
  {\it ``Higgs boson production in weak boson fusion at next-to-leading order,''} \\
  Phys.\ Rev.\ D {\bf 70}, 073011 (2004)
  [arXiv:hep-ph/0403194].
\item J.~Campbell, R.~K.~Ellis and F.~Tramontano, \\
  {\it ``Single top production and decay at next-to-leading order,''} \\
  Phys.\ Rev.\ D {\bf 70}, 094012 (2004)
  [arXiv:hep-ph/0408158].
\item J.~Campbell and F.~Tramontano, \\
  {\it ``Next-to-leading order corrections to Wt production and
  decay,''} \\
  Nucl.\ Phys.\ B {\bf 726}, 109 (2005)
  [arXiv:hep-ph/0506289].
\item J.~Campbell, R.K.~Ellis, F.~Maltoni and S.~Willenbrock, \\
  {\it ``Production of a $Z$ boson and two jets with one heavy
  quark tag,''} \\
  Phys.\ Rev.\ D {\bf 73}, 054007 (2006)
  [arXiv:hep-ph/0510362].
\end{itemize}

The following recent publications have also made use of calculations
implemented in MCFM, but the corresponding code has not yet been made
public. Versions of the code that contain these calculations will be
released in the future.

\begin{itemize}
\item J.~M.~Campbell, R.~K.~Ellis and G.~Zanderighi, \\
  {\it ``Next-to-leading order Higgs~$+~2$~jet production via
  gluon fusion,''} \\
  JHEP {\bf 0610}, 028 (2006)
  [arXiv:hep-ph/0608194].
\item J.~Campbell, R.K.~Ellis, F.~Maltoni and S.~Willenbrock, \\
  {\it ``Production of a $W$ boson and two jets with one $b$-quark
  tag,''} \\
  Phys.\ Rev.\ D {\bf 75}, 054015 (2007)
  [arXiv:hep-ph/0611348].
\item J.~Campbell, F.~Maltoni and F.~Tramontano, \\
  {\it ``QCD corrections to $J/\psi$ and $\Upsilon$ production
  at hadron colliders,''} \\
  Phys. Rev. Lett. {\bf 98}, 252002 (2007)
  [arXiv:hep-ph/0703113].
\item R.~K.~Ellis, K.~Melnikov and G.~Zanderighi,\\
  {\it ``Generalized unitarity at work: first NLO QCD results for hadronic $W+$~3 jet production,''}
  arXiv:0901.4101 [hep-ph].
\item J.~M.~Campbell, R.~Frederix, F.~Maltoni and F.~Tramontano,\\
  {\it $t$-channel single-top production at hadron colliders,''}
  arXiv:0903.0005 [hep-ph].
\end{itemize}

\section{Installation}

The tar'ed and gzip'ed package may be downloaded from
the {\tt MCFM} home-page at {\tt http://mcfm.fnal.gov}.
After extracting, the source can be initialized by running the
{\tt Install} command and then compiled with {\tt make}. The
{\tt Install} script may be edited prior to running, to include
the locations of the CERNLIB and LHAPDF libraries, if desired.
The code has been developed and tested under Redhat Linux. Please report
any compilation problems under other operating systems to the authors.

The directory structure of the installation is as follows:
\begin{itemize}
\item {\tt Doc}. The source for this document.
\item {\tt Bin}. The directory containing the executable {\tt mcfm},
and various essential files -- notably the options file {\tt input.DAT}.
\item {\tt Bin/Pdfdata}. The directory containing the PDF data-files.
\item {\tt obj}. The object files produced by the compiler. 
\item {\tt src}. The Fortran source files in various subdirectories.
\end{itemize}
The files which it is most likely that the user will need to modify
are located in {\tt src/User}. It is convenient, if one wants to 
modify one of these files, (or any other file in the subdirectories of the 
{\tt src} directory),
to copy it first to the directory where the user has installed {\tt mcfm}.
The makefile will use this file in preference to the identically named
file in the sub-directories of {\tt src}.
 
\section{Input parameters}

{\tt MCFM} now allows the user to choose between a number of schemes
for defining the electroweak couplings. These choices are summarized
in Table~\ref{ewscheme}. The scheme is selected by modifying the
value of {\tt ewscheme} in {\tt src/User/mdata.f} prior to compilation, 
which also contains
the values of all input parameters (see also Table~\ref{default}).

\begin{table}
\begin{center}
\begin{tabular}{|c|c|c|c|c|c|c|} \hline
 Paramter & Name & Input Value
 & \multicolumn{4}{c|}{Output Value determined by \tt ewscheme} \\
\cline{4-7}
& ({\tt \_inp}) & & {\tt -1} & {\tt 0} & {\tt 1} & {\tt 2} \\ \hline
$G_F$            & {\tt Gf}      & 1.16639$\times$10$^{-5}$ 
 & input & calculated & input & input \\
$\alpha(M_Z)$    & {\tt aemmz}   & 1/128.89                 
 & input & input & calculated & input \\
$\sin^2 \theta_w$& {\tt xw}      & 0.2312               
 & calculated & input & calculated & input \\
$M_W$            & {\tt wmass}   & 80.419 GeV                
 & input & calculated & input & calculated \\
$M_Z$            & {\tt zmass}   & 91.188 GeV               
 & input & input & input & calculated \\
$m_t$            & {\tt mt}      & 172.5 GeV                  
 & calculated & input & input & input \\
\hline
\end{tabular}
\caption{Different options for the scheme used to fix the electroweak
parameters of the Standard Model and the corresponding default input
values.}
\label{ewscheme}
\end{center}
\end{table}

Starting from version 5.2 of the code, the default scheme has been
changed from {\tt ewscheme=-1} (as in previous versions) to
{\tt ewscheme=+1}. As described below, this corresponds to a scheme
in which the top quark mass is an input parameter so that it is
more suitable for many processes now included in the program.

The choice of ({\tt ewscheme=-1}) enforces the use of an effective field
theory approach, which is valid for scales below the top mass. In this
approach there are 4 independent parameters (which we choose to be
$G_F$, $\alpha(M_Z)$, $M_W$ and $M_Z$). For further details,
see Georgi~\cite{Georgi:1991ci}.

For all the other schemes ({\tt ewscheme=0,1,2}) the top mass is simply
an additional input parameter and there are 3 other independent
parameters from the remaining 5. The variable {\tt ewscheme} then performs
exactly the same role as {\tt idef} in MadEvent~\cite{Maltoni:2002qb}.
{\tt ewscheme=0} is the old MadEvent default and {\tt ewscheme=1} is the
new MadEvent default, which is also the same as that used in 
Alpgen~\cite{Alpgen} and LUSIFER~\cite{Lusifer} 
For processes in which the top quark is directly produced  it is 
preferable to use  the schemes ({\tt ewscheme=0,1,2}), since in these schemes
one can adjust the top mass to its physical value, (in {\tt src/User/mdata.f}).

\begin{table}
\begin{center}
\begin{tabular}{|c|c|c|} \hline
Parameter & Fortran name & Default value \\ 
\hline
$m_\tau$         & {\tt mtau}      & 1.777 GeV            \\
$m^2_\tau$& {\tt mtausq}  & 3.1577 GeV$^2$     \\
$m^2_c$   & {\tt mcsq}    & 2.25  GeV$^2$       \\
$m^2_b$   & {\tt mbsq}    & 17.64 GeV$^2$       \\
$\Gamma_\tau$    & {\tt tauwidth}& 2.269$\times$10$^{-12}$~GeV \\
$\Gamma_W$       & {\tt wwidth}  & 2.06 GeV               \\
$\Gamma_Z$       & {\tt zwidth}  & 2.49 GeV               \\
$V_{ud}$         & {\tt Vud}     & 0.975                  \\
$V_{us}$         & {\tt Vus}     & 0.222             \\
$V_{ub}$         & {\tt Vub}     & 0.                     \\
$V_{cd}$         & {\tt Vcd}     & 0.222             \\
$V_{cs}$         & {\tt Vcs}     & 0.975                  \\
$V_{cb}$         & {\tt Vcb}     & 0.                     \\
\hline
\end{tabular}
\caption{Default values for the remaining parameters in {\tt MCFM}.}
\label{default} 
\end{center}
\end{table}

In the same file ({\tt mdata.f}) one can also choose the definition
that the program uses for computing transverse quantities, namely
transverse momentum or transverse energy. These are defined by,
\begin{eqnarray}
\mbox{transverse momentum:} & \sqrt{p_x^2+p_y^2} \; ,\nonumber \\
\mbox{transverse energy:}   &
 \frac{E \sqrt{p_x^2+p_y^2}}{\sqrt{p_x^2+p_y^2+p_z^2}} \; .
\end{eqnarray}
The two definitions of course coincide for massless particles.
The chosen definition is used for all cuts that are applied to the
process and it is the one that is used in the default set of histograms.

\subsection{Parton distributions}
The value of $\alpha_S(M_Z)$ is not adjustable; it is hardwired with the
parton distribution. In addition, the parton distribution also specifies
the number of loops that should be used in the running of $\alpha_S$.
The default mode of operation is to choose from a
collection of modern parton distribution functions that are included with
MCFM.  The distributions, together with their associated $\alpha_S(M_Z)$
values, are given in Table\ref{pdlabel}. 
For the older distributions, where the
coupling was specified by $\Lambda$ this requires 
some calculation and/or guesswork.
\begin{table}[h]
\begin{center}
\begin{tabular}{|c|c|c||c|c|c|}
\hline
{\tt mstw8lo}  & 0.1394       & \mstwoheight &
{\tt mstw8nl}  & 0.1202       & \mstwoheight \\
{\tt mstw8nn}  & 0.1171       & \mstwoheight &
{\tt mrs02nl}  & 0.1197       & \mrstohtwo \\
{\tt mrs02nn}  & 0.1154       & \mrstohtwo &
{\tt mrs4nf3}  & 0.1083       & \mrstff \\
{\tt mrs4lf3}  & 0.1186       & \mrstff &
{\tt mrs4nf4}  & 0.1153       & \mrstff \\
{\tt mrs4lf4}  & 0.1251       & \mrstff &
{\tt mrs0119}  & 0.119        & \mrstohone \\
{\tt mrs0117}  & 0.117        & \mrstohone &
{\tt mrs0121}  & 0.121        & \mrstohone \\
{\tt mrs01\_j} & 0.121        & \mrstohone &
{\tt mrs99\_1} & 0.1175       & \mrsninenine \\
{\tt mrs99\_2} & 0.1175       & \mrsninenine &
{\tt mrs99\_3} & 0.1175       & \mrsninenine \\
{\tt mrs99\_4} & 0.1125       & \mrsninenine &    
{\tt mrs99\_5} & 0.1225       & \mrsninenine \\
{\tt mrs99\_6} & 0.1178       & \mrsninenine &    
{\tt mrs99\_7} & 0.1171       & \mrsninenine \\
{\tt mrs99\_8} & 0.1175       & \mrsninenine &    
{\tt mrs99\_9} & 0.1175       & \mrsninenine \\
{\tt mrs9910}  & 0.1175       & \mrsninenine &    
{\tt mrs9911}  & 0.1175       & \mrsninenine \\
{\tt mrs9912}  & 0.1175       & \mrsninenine &    
{\tt mrs98z1}  &  0.1175      & \mrsnineeight \\  
{\tt mrs98z2}  &  0.1175      & \mrsnineeight & 
{\tt mrs98z3}  &  0.1175      & \mrsnineeight \\  
{\tt mrs98z4}  &  0.1125      & \mrsnineeight &  
{\tt mtungb1}  &  0.109       & \mrsnineeight \\
{\tt mrs98z5}  &  0.1225      & \mrsnineeight &   
{\tt mrs96r1}  &  0.113       & \mrsninesix \\    
{\tt mrs96r2}  &  0.120       & \mrsninesix &  
{\tt mrs96r3}  &  0.113       & \mrsninesix \\   
{\tt mrs96r4}  &  0.120       & \mrsninesix &   
{\tt mrs95ap}  &  0.1127      & \mrsninefive \\
{\tt mrs95\_g} &  0.1148      & \mrsninefive &
{\tt hmrs90e}  &  0.09838     & \hmrs \\ 
{\tt hmrs90b}  &  0.10796     & \hmrs &
{\tt cteq66m}  &  0.118       & \cteqsixsixm \\
{\tt cteq61m}  &  0.118       & \cteqsixonem &
{\tt cteq6\_m} &  0.118       & \cteqsix \\
{\tt cteq6\_d} &  0.118       & \cteqsix &
{\tt cteq6\_l} &  0.118       & \cteqsix \\
{\tt cteq6l1}  &  0.130       & \cteqsix &
{\tt cteq5hq}  &  0.118       & \cteqfive \\
{\tt cteq5f3}  &  0.106       & \cteqfive &
{\tt cteq5f4}  &  0.112       & \cteqfive \\
{\tt cteq5\_m} &  0.118       & \cteqfive &
{\tt cteq5\_d} &  0.118       & \cteqfive \\ 
{\tt cteq5\_l} &  0.127       & \cteqfive & 
{\tt cteq5l1}  &  0.127       & \cteqfive \\ 
{\tt cteq5hj}  &  0.118       & \cteqfive &
{\tt cteq5m1}  &  0.118       & \cteqfive \\
{\tt ctq5hq1}  &  0.118       & \cteqfive &
{\tt cteq4a5}  &  0.122       & \cteqfour \\
{\tt cteq4hj}  &  0.116       & \cteqfour &
{\tt cteq4lq}  &  0.114       & \cteqfour \\
{\tt cteq4\_m} &  0.116       & \cteqfour &
{\tt cteq4\_d} &  0.116       & \cteqfour \\
{\tt cteq4\_l} &  0.132       & \cteqfour &
{\tt cteq4a1}  &  0.110       & \cteqfour \\
{\tt cteq4a2}  &  0.113       & \cteqfour &
{\tt cteq4a3}  &  0.116       & \cteqfour \\
{\tt cteq4a4}  &  0.119       & \cteqfour &
{\tt cteq3\_m} &  0.112       & \cteqthree \\
{\tt cteq3\_l} &  0.112       & \cteqthree &
{\tt cteq3\_d} &  0.112       & \cteqthree \\
\hline
\end{tabular}
\end{center}
\caption{Available pdf sets, their corresponding values of
$\alpha_S(M_Z)$ and a reference to the paper or preprint that
describes their origin.
\label{pdlabel}}
\end{table}

By editing the {\tt Makefile}, it is straightforward to switch to
either the {\tt PDFLIB} or the {\tt LHAPDF} parton distribution
function implementations.

To use {\tt PDFLIB}, one must first set the variable {\tt CERNLIB}
in the makefile to point to the directory that contains
{\tt libpdflib804.a} and then modify {\tt PDFROUTINES} to
take the value {\tt PDFLIB}. The parameters to choose the
pdf set are then specified in {\tt Bin/input.DAT}.

To use {\tt LHAPDF}, one must first set the variable {\tt LHAPDFLIB}
in the makefile to point to the directory that contains
{\tt libLHAPDF.a} and then modify {\tt PDFROUTINES} to
take the value {\tt LHAPDF}. Note that, in newer versions of LHAPDF,
it may be easier to link against the static LHAPDF libraries by passing
a ``{\tt -static}'' flag to the compiler via the {\tt FFLAGS} variable
in the makefile, rather than using the (default) shared libraries.
Particularly when compiling against both LHAPDF and CERNLIB, it may
be useful to link only LHAPDF in a static manner. This can be achieved
using g77, for instance, by replacing
 ``{\tt -lLHAPDF}'' with ``{\tt -Wl,-Bstatic -lLHAPDF -Wl,-Bdynamic}''
 in the makefile. This version of {\tt MCFM} has been explicitly tested
 against {\tt LHAPDF-5.5.0}.
 
The parameters to choose the
pdf set are then provided in {\tt Bin/input.DAT} - 
the name of the group and the integer specifying 
the set.
{\tt MCFM} expects to find the sets in a sub-directory of {\tt Bin} called
{\tt PDFsets}, as in the {\tt LHAPDF} distribution. It is easiest to
simply create a symbolic link appropriately.

One may always return to the built-in distributions by resetting
{\tt PDFROUTINES} to take the value {\tt NATIVE}  in the makefile,
(and recompiling).
\clearpage
\section{Runtime options}

{\tt mcfm} execution is performed in the {\tt Bin/} directory,
with syntax:
\begin{center}
{\tt mcfm [}{\it mydir}{\tt ] [}{\it myfile}{\tt .DAT]}
\end{center}
The executable {\tt mcfm} is automatically moved to {\tt Bin} by the makefile.
If no command line options are given, then {\tt mcfm} will default
to using the file {\tt input.DAT} in the current directory for
choosing options\footnote{Note that this is very different from
previous versions of {\tt MCFM}. All auxiliary input files from v3.2 and
earlier have now been incorporated into a single file.}.
The different possibilities are summarized in Table~\ref{clopts}.
\begin{table}
\begin{center}
\begin{tabular}{l|cl}
Command executed && Location of input file \\
\hline
{\tt mcfm}                      && {\tt input.DAT} \\
{\tt mcfm myfile.DAT}           && {\tt myfile.DAT} \\
{\tt mcfm mydir}                && {\tt mydir/input.DAT} \\
{\tt mcfm mydir myfile.DAT}     && {\tt mydir/myfile.DAT} \\
\end{tabular}
\end{center}
\caption{Summary of command line options for running {\tt mcfm}.}
\label{clopts}
\end{table}
In addition, if a working directory {\it mydir} is specified then
output files will also be produced in this directory. By using these
options one may, for instance, keep all input and output files for
different processes in separate directories.

Each parameter in the input file is specified by a line such as
\begin{displaymath}
{\tt value} \hspace{3cm} {\tt [parameter]}
\end{displaymath}
and we will give a description of all the parameters below, together with
valid and/or sensible inputs for ${\tt value}$. Groups of parameters
are separated by a blank line and a description of that section, for
readability.

\begin{itemize}
\item {\tt file version number}. This should match the version number
that is printed when {\tt mcfm} is executed.

\begin{center}
\{blank line\} \\
{\tt [Flags to specify the mode in which MCFM is run] }
\end{center}

\item {\tt evtgen}. The default for this, and the following three
parameters, is {\tt .false.} and this corresponds to the usual mode
of operation. It is possible to generate n-tuples instead of histograms,
as well as unweighted events, for some processes. Please refer to
Section~\ref{subsec:otheroutput} for further details.
\item {\tt creatent}. {\it See above.}
\item {\tt skipnt}. {\it See above.}
\item {\tt dswhisto}. {\it See above.}

\begin{center}
\{blank line\} \\
{\tt [General options to specify the process and execution] }
\end{center}

\item {\tt nproc}.
The process to be studied is given by
choosing a process number, according to Table~\ref{nproctable}.
$f(p_i)$ denotes a generic partonic jet.

\begin{table}
\begin{center}
\hspace*{-1.5cm}
\begin{tabular}{|l|l|l|}
\hline
{\tt nproc} & $f(p_1)+f(p_2) \to \ldots $& Order \\
\hline
 
1 & $ W^+(\to \nu(p_3)+e^+(p_4)) $ & NLO \\
6 & $ W^-(\to e^-(p_3)+\bar{\nu}(p_4)) $ & NLO \\
\hline 
11& $ W^+(\to \nu(p_3)+e^+(p_4))+f(p_5) $ & NLO \\
12& $ W^+(\to \nu(p_3)+e^+(p_4))+\gamma(p_5) $ & NLO \\
13& $ W^+(\to \nu(p_3)+e^+(p_4))+\bar{c}(p_5) $ & NLO \\
14& $ W^+(\to \nu(p_3)+e^+(p_4))+\bar{c}(p_5) ~[ \mbox{massless} ] $ & NLO \\
16& $ W^-(\to e^-(p_3)+\bar{\nu}(p_4))+f(p_5) $ & NLO \\
17& $ W^-(\to e^-(p_3)+\bar{\nu}(p_4))+\gamma(p_5) $ & NLO \\
18& $ W^-(\to e^-(p_3)+\bar{\nu}(p_4))+c(p_5) $ & NLO \\
19& $ W^-(\to e^-(p_3)+\bar{\nu}(p_4))+c(p_5) ~[ \mbox{massless} ] $ & NLO \\
\hline 
20& $ W^+(\to \nu(p_3)+e^+(p_4)) +b(p_5)+\bar{b}(p_6) ~[ \mbox{massive} ] $ & LO \\
21& $ W^+(\to \nu(p_3)+e^+(p_4)) +b(p_5)+\bar{b}(p_6) $ & NLO \\
22& $ W^+(\to \nu(p_3)+e^+(p_4)) +f(p_5)+f(p_6) $ & NLO \\
23& $ W^+(\to \nu(p_3)+e^+(p_4)) +f(p_5)+f(p_6)+f(p_7) $ & LO \\
24& $ W^+(\to \nu(p_3)+e^+(p_4)) +b(p_5)+\bar{b}(p_6)+f(p_7) $ & LO \\
25& $ W^-(\to e^-(p_3)+\bar{\nu}(p_4)) +b(p_5)+\bar{b}(p_6) ~[ \mbox{massive} ] $ & LO \\
26& $ W^-(\to e^-(p_3)+\bar{\nu}(p_4)) +b(p_5)+\bar{b}(p_6) $ & NLO \\
27& $ W^-(\to e^-(p_3)+\bar{\nu}(p_4)) +f(p_5)+f(p_6) $ & NLO \\
28& $ W^-(\to e^-(p_3)+\bar{\nu}(p_4)) +f(p_5)+f(p_6)+f(p_7) $ & LO \\
29& $ W^-(\to e^-(p_3)+\bar{\nu}(p_4)) +b(p_5)+\bar{b}(p_6)+f(p_7) $ & LO \\
\hline 
31& $ Z^0(\to e^-(p_3)+e^+(p_4)) $ & NLO \\
32& $ Z^0(\to 3\times(\nu(p_3)+\bar{\nu}(p_4))) $ & NLO \\
33& $ Z^0(\to b(p_3)+\bar{b}(p_4)) $ & NLO \\
\hline 
41& $ Z^0(\to e^-(p_3)+e^+(p_4))+f(p_5) $ & NLO \\
42& $ Z_0(\to 3\times(\nu(p_3)+\bar{\nu}(p_4)))-~[ \mbox{sum over 3 $\nu$} ]+f(p_5) $ & NLO \\
43& $ Z^0(\to b(p_3)+\bar{b}(p_4))+f(p_5) $ & NLO \\
\hline 
44& $ Z^0(\to e^-(p_3)+e^+(p_4))+f(p_5)+f(p_6) $ & NLO \\
45& $ Z^0(\to e^-(p_3)+e^+(p_4))+f(p_5)+f(p_6)+f(p_7) $ & LO \\
\hline 
48& $ Z^0(\to e^-(p_3)+e^+(p_4))+\gamma(p_5) $ & NLO \\
49& $ Z^0(\to 3\times(\nu(p_3)+\bar{\nu}(p_4)))-~[ \mbox{sum over 3 $\nu$} ]+\gamma(p_5) $ & NLO \\
\hline 
50& $ Z^0(\to e^-(p_3)+e^+(p_4))+\bar{b}(p_5)+b(p_6) ~[ \mbox{massive} ] $ & LO \\
51& $ Z^0(\to e^-(p_3)+e^+(p_4))+b(p_5)+\bar{b}(p_6) $ & NLO \\
52& $ Z_0(\to 3\times(\nu(p_3)+\bar{\nu}(p_4)))+b(p_5)+\bar{b}(p_6) $ & NLO \\
53& $ Z^0(\to b(p_3)+\bar{b}(p_4))+b(p_5)+\bar{b}(p_6) $ & NLO \\
54& $ Z^0(\to e^-(p_3)+e^+(p_4))+b(p_5)+\bar{b}(p_6)+f(p_7) $ & LO \\
\hline 
56& $ Z^0(\to e^-(p_3)+e^+(p_4))+c(p_5)+\bar{c}(p_6) $ & NLO \\
\hline 
\end{tabular}
\end{center}
\end{table}
\begin{table}
\begin{center}
\hspace*{-1.5cm}
\begin{tabular}{|l|l|l|}
\hline
{\tt nproc} & $ f(p_1)+f(p_2) \to \ldots $& Order \\ 
\hline
61& $ W^+(\to \nu(p_3)+e^+(p_4)) +W^-(\to e^-(p_5)+\bar{\nu}(p_6)) $ & NLO \\
62& $ W^+(\to \nu(p_3)+e^+(p_4)) +W^-(\to q(p_5)+\bar{q}(p_6)) $ & NLO \\
63& $ W^+(\to  q(p_3)+ \bar{q}(p_4)) +W^-(\to e^-(p_5)+\bar{\nu}(p_6)) $ & NLO \\
64& $ W^+(\to \nu(p_3)+e^+(p_4)) +W^-(\to e^-(p_5)+\bar{\nu}(p_6)) ~[ \mbox{no pol} ] $ & NLO \\
66& $ W^+(\to  q(p_3)+ \bar{q}(p_4)) +W^-(\to e^-(p_5)+\bar{\nu}(p_6))+f(p_7) $ & NLO \\
\hline 
71& $ W^+(\to \nu(p_3)+\mu^+(p_4))+Z^0(\to e^-(p_5)+e^+(p_6)) $ & NLO \\
72& $ W^+(\to \nu(p_3)+\mu^+(p_4))+Z^0(\to \nu_e(p_5)+\bar{\nu}_e(p_6)) $ & NLO \\
73& $ W^+(\to \nu(p_3)+\mu^+(p_4))+Z^0(\to b(p_5)+\bar{b}(p_6)) $ & NLO \\
\hline 
76& $ W^-(\to \mu^-(p_3)+\bar{\nu}(p_4))+Z^0(\to e^-(p_5)+e^+(p_6)) $ & NLO \\
77& $ W^-(\to e^-(p_3)+\bar{\nu}(p_4))+Z^0(\to \nu(p_5)+\bar{\nu}(p_6)) $ & NLO \\
78& $ W^-(\to e^-(p_3)+\bar{\nu}(p_4))+Z^0(\to b(p_5)+\bar{b}(p_6)) $ & NLO \\
\hline 
81& $ Z^0(\to \mu^-(p_3)+\mu^+(p_4)) + Z^0(\to e^-(p_5)+e^+(p_6)) $ & NLO \\
82& $ Z^0(\to e^-(p_3)+e^+(p_4)) + Z^0(\to 3\times(\nu(p_5)+\bar{\nu}(p_6)))$ & NLO \\
83& $ Z^0(\to e^-(p_5)+e^+(p_6)) + Z^0(\to b(p_3)+\bar{b}(p_4)) $ & NLO \\
84& $ Z^0(\to b(p_3)+\bar{b}(p_4)) + Z^0(\to 3\times(\nu(p_5)+\bar{\nu}(p_6)))$ & NLO \\
85& $ Z^0(\to e^-(p_3)+e^+(p_4)) + Z^0(\to 3\times(\nu(p_5)+\bar{\nu}(p_6))) + f(p_7)$ & LO \\
\hline 
86& $ Z^0(\to \mu^-(p_3)+\mu^+(p_4)) + Z^0(\to e^-(p_5)+e^+(p_6))~[ \mbox{no $\gamma^*$} ] $ & NLO \\
87& $ Z^0(\to e^-(p_3)+e^+(p_4))+Z^0(\to 3\times(\nu(p_5)+\bar{\nu}(p_6))) ~[ \mbox{no $\gamma^*$} ] $ & NLO \\
88& $ Z^0(\to e^-(p_3)+e^+(p_4))+Z^0(\to b(p_5)+\bar{b}(p_6)) ~[ \mbox{no $\gamma^*$} ] $ & NLO \\
89& $ Z^0(\to b(p_3)+\bar{b}(p_4))+Z^0(\to 3\times(\nu(p_5)+\bar{\nu}(p_6)))~[ \mbox{no $\gamma^*$} ] $ & NLO \\
\hline 
91& $ W^+(\to \nu(p_3)+e^+(p_4)) + H(\to b(p_5)+\bar{b}(p_6))  $ & NLO \\
92& $ W^+(\to \nu(p_3)+e^+(p_4)) + H(\to W^+(\nu(p_5) e^+(p_6)) + W^-(e^-(p_7) \bar{\nu}(p_8)))  $ & NLO \\
96& $ W^-(\to e^-(p_3)+\bar{\nu}(p_4)) + H(\to b(p_5)+\bar{b}(p_6)) $ & NLO \\ 
97& $ W^-(\to e^-(p_3)+\bar{\nu}(p_4)) + H(\to W^+(\nu(p_5) e^+(p_6)) + W^-(e^-(p_7) \bar{\nu}(p_8)))  $ & NLO \\
\hline 
101& $ Z^0(\to e^-(p_3)+e^+(p_4)) + H(\to b(p_5)+\bar{b}(p_6)) $ & NLO \\
102& $ Z^0(\to 3\times(\nu(p_3)+\bar{\nu}(p_4))) + H(\to b(p_5)+\bar{b}(p_6)) $ & NLO \\
103& $ Z^0(\to b(p_3)+\bar{b}(p_4)) + H(\to b(p_5)+\bar{b}(p_6)) $ & NLO \\     
106& $ Z^0(\to e^-(p_3)+e^+(p_4)) + H(\to W^+(\nu(p_5) e^+(p_6)) + W^-(e^-(p_7) \bar{\nu}(p_8))) $ & NLO \\
107& $ Z^0(\to 3\times(\nu(p_3)+\bar{\nu}(p_4))) + H(\to W^+(\nu(p_5) e^+(p_6)) + W^-(e^-(p_7) \bar{\nu}(p_8))) $ & NLO \\
108& $ Z^0(\to b(p_3)+\bar{b}(p_4)) + H(\to W^+(\nu(p_5) e^+(p_6)) + W^-(e^-(p_7) \bar{\nu}(p_8))) $ & NLO \\     
\hline 
111& $ H(\to b(p_3)+\bar{b}(p_4)) $ & NLO \\
112& $ H(\to \tau^-(p_3)+\tau^+(p_4)) $ & NLO \\
113& $ H(\to  W^+(\nu(p_3)+e^+(p_4)) + W^-(e^-(p_5)+\bar{\nu}(p_6))) $ & NLO \\
114& $ H(\to Z^0(\mu^-(p_3)+\mu^+(p_4)) + Z^0(e^-(p_5)+e^+(p_6)) $ & NLO \\
115& $ H(\to Z^0(3\times(\nu(p_3)+\bar{\nu}(p_4)))+ Z^0(e^-(p_5)+e^+(p_6)) $ & NLO \\
116& $ H(\to Z^0(\mu^-(p_3)+\mu^+(p_4)) + Z^0(b(p_5)+\bar{b}(p_6)) $ & NLO \\
\hline 
\end{tabular}
\end{center}
\end{table}
\begin{table}
\begin{center}
\hspace*{-1.5cm}
\begin{tabular}{|l|l|l|}
\hline
{\tt nproc} & $ f(p_1)+f(p_2) \to \ldots $& Order \\ 
\hline
141& $ H(\to b(p_3)+\bar{b}(p_4)) + b(p_5) (+g(p_6)) $ & NLO \\
142& $ H(\to b(p_3)+\bar{b}(p_4)) + \bar{b}(p_5) (+b(p_6)) $ & NLO \\
143& $ H(\to b(p_3)+\bar{b}(p_4)) + b(p_5) + \bar{b}(p_6) ~[ \mbox{both observed} ] $ & NLO \\
\hline 
151& $ t(\to \nu(p_3)+e^+(p_4)+b(p_5))+\bar{t}(\to \bar{b}(p_6)+e^-(p_7)+\bar{\nu}(p_8)) $ & LO \\
152& $ t(\to \nu(p_3)+e^+(p_4)+b(p_5))+\bar{t}(\to \bar{b}(p_6)+q(p_7)+\bar{q}(p_8)) $ & LO \\
\hline 
156& $t(\to \nu(p_3)+e^+(p_4)+b(p_5))+\bar{t}(\to \bar{b}(p_6) +e^-(p_7)+\bar{\nu}(p_8))+g(p_9) $ & LO \\
\hline 
157& $ t \bar{t} ~[ \mbox{for total Xsect} ] $ & NLO \\
158& $ b \bar{b} ~[ \mbox{for total Xsect} ] $ & NLO \\
159& $ c \bar{c} ~[ \mbox{for total Xsect} ] $ & NLO \\
160& $ t \bar{t} + g ~[ \mbox{for total Xsect} ] $ & LO \\
\hline 
161& $ t(\to \nu(p_3)+e^+(p_4)+b(p_5))+q(p_6) ~[ \mbox{t-channel} ] $ & NLO \\
162& $ t(\to \nu(p_3)+e^+(p_4)+b(p_5))+q(p_6) ~[ \mbox{decay} ] $ & NLO \\
166& $ \bar{t}(\to e^-(p_3)+\bar{\nu}(p_4)+\bar{b}(p_5))+q(p_6) ~[ \mbox{t-channel} ] $ & NLO \\
167& $ \bar{t}(\to e^-(p_3)+\bar{\nu}(p_4)+\bar{b}(p_5))+q(p_6) ~[ \mbox{decay} ] $ & NLO \\
\hline 
171& $ t(\to \nu(p_3)+e^+(p_4)+b(p_5))+\bar{b}(p_6)) ~[ \mbox{s-channel} ] $ & NLO \\
172& $ t(\to \nu(p_3)+e^+(p_4)+b(p_5))+\bar{b}(p_6)) ~[ \mbox{decay} ] $ & NLO \\
176& $ \bar{t}(\to e^-(p_3)+\bar{\nu}(p_4)+\bar{b}(p_5))+b(p_6)) ~[ \mbox{s-channel} ] $ & NLO \\
177& $ \bar{t}(\to e^-(p_3)+\bar{\nu}(p_4)+\bar{b}(p_5))+b(p_6)) ~[ \mbox{decay} ] $ & NLO \\
\hline 
180& $ W^-(\to e^-(p_3)+\bar{\nu}(p_4))+t(p_5) $ & NLO \\
181& $ W^-(\to e^-(p_3)+\bar{\nu}(p_4))+t(\nu(p_5)+e^+(p_6)+b(p_7)) $ & NLO \\
182& $ W^-(\to e^-(p_3)+\bar{\nu}(p_4))+t(\nu(p_5)+e^+(p_6)+b(p_7)) ~[ \mbox{decay} ] $ & NLO \\
185& $ W^+(\to \nu(p_3)+e^+(p_4))+\bar{t}(p_5) $ & NLO \\
186& $ W^+(\to \nu(p_3)+e^+(p_4))+\bar{t}(e^-(p_5)+\bar{\nu}(p_6)+\bar{b}(p_7)) $ & NLO \\
187& $ W^+(\to \nu(p_3)+e^+(p_4))+\bar{t}(e^-(p_5)+\bar{\nu}(p_6)+\bar{b}(p_7)) ~[ \mbox {decay} ] $ & NLO \\
\hline 
190& $t(p_3)+\bar{t}(p_4)+H(p_5) $ & LO \\
191& $ t(\to \nu(p_3)+e^+(p_4)+b(p_5))+\bar{t}(\to \bar{\nu}(p_7)+e^-(p_8)+\bar{b}(p_6))+H(p_9+p_{10}) $ & LO \\
\hline 
196& $ t(\to \nu(p_3)+e^+(p_4)+b(p_5))+\bar{t}(\to \bar{\nu}(p_7)+e^-(p_8)+\bar{b}(p_6))+Z(e^-(p_9),e^+(p_{10})) $ & LO \\
197& $ t(\to \nu(p_3)+e^+(p_4)+b(p_5))+\bar{t}(\to \bar{\nu}(p_7)+e^-(p_8)+\bar{b}(p_6))+Z(b(p_9),\bar{b}(p_{10})) $ & LO \\
\hline 
201& $ H(\to b(p_3)+\bar{b}(p_4)) + f(p_5) ~[ \mbox{full mt dep.} ] $ & LO \\
202& $ H(\to \tau^-(p_3)+\tau^+(p_4)) + f(p_5) ~[ \mbox{full mt dep.} ] $ & LO \\
203& $ H(\to b(p_3)+\bar{b}(p_4)) + f(p_5) $ & NLO \\
204& $ H(\to \tau^-(p_3)+\tau^+(p_4)) + f(p_5) $ & NLO \\
206& $ A(\to b(p_3)+\bar{b}(p_4)) + f(p_5) ~[ \mbox{full mt dep.} ] $ & LO \\
207& $ A(\to \tau^-(p_3)+\tau^+(p_4)) + f(p_5) ~[ \mbox{full mt dep.} ] $ & LO \\
208& $ H(\to W^+(\to \nu(p_3)+e^+(p_4))+W^-(\to e^-(p_5)+\bar{\nu}(p_6)))+f(p_7)$ & NLO \\
\hline 
\end{tabular}
\end{center}
\end{table}
\begin{table}
\begin{center}
\hspace*{-1.5cm}
\begin{tabular}{|l|l|l|}
\hline
{\tt nproc} & $ f(p_1)+f(p_2) \to \ldots $& Order \\ 
\hline 
211& $ H(\to b(p_3)+\bar{b}(p_4))+f(p_5)+f(p_6) ~[ \mbox{WBF} ] $ & NLO \\
212& $ H(\to \tau^-(p_3)+\tau^+(p_4))+f(p_5)+f(p_6) ~[ \mbox{WBF} ] $ & NLO \\
213& $ H(\to W^+(\nu(p_3) e^+(p_4)) + W^-(e^-(p_5) \bar{\nu}(p_6)))+f(p_7)+f(p_8) ~[ \mbox{WBF} ] $ & NLO \\
216& $ H(\to b(p_3)+\bar{b}(p_4))+f(p_5)+f(p_6)+f(p_7) ~[ \mbox{WBF+jet} ] $ & NLO \\
217& $ H(\to \tau^-(p_3)+\tau^+(p_4))+f(p_5)+f(p_6)+f(p_7) ~[ \mbox{WBF+jet} ] $ & NLO \\
\hline 
221& $ \tau^-(\to e^-(p_3)+\bar{\nu}_e(p_4)+\nu_\tau(p_5))+\tau^+(\to \bar{\nu}_\tau(p_6)+\nu_e(p_7)+e^+(p_8)) $ & LO \\
\hline
261& $ Z^0(\to e^-(p_3)+e^+(p_4))+b(p_5) $ & NLO \\
262& $ Z^0(\to e^-(p_3)+e^+(p_4))+c(p_5) $ & NLO \\
263& $ Z^0(\to e^-(p_3)+e^+(p_4))+\bar{b}(p_5)+b(p_6) ~[ \mbox{1 b-tag} ] $ & NLO \\
264& $ Z^0(\to e^-(p_3)+e^+(p_4))+\bar{c}(p_5)+c(p_6) ~[ \mbox{1 c-tag} ] $ & NLO \\
266& $ Z^0(\to e^-(p_3)+e^+(p_4))+b(p_5)(+\bar{b}(p_6)) $ & NLO \\
267& $ Z^0(\to e^-(p_3)+e^+(p_4))+c(p_5)(+\bar{c}(p_6)) $ & NLO \\
\hline 
271& $ H(b(p_3)+\bar{b}(p_4))+f(p_5)+f(p_6)~[ \mbox{in heavy top limit} ] $ & LO \\
272& $ H(\tau^-(p_3)+\tau^+(p_4))+f(p_5)+f(p_6)~[ \mbox{in heavy top limit} ] $ & LO \\
273& $ H(b(p_3)+\bar{b}(p_4))+f(p_5)+f(p_6)+f(p_7)~[ \mbox{in heavy top limit} ] $ & LO \\
274& $ H(\tau^-(p_3)+\tau^+(p_4))+f(p_5)+f(p_6)+f(p_7)~[ \mbox{in heavy top limit} ] $ & LO \\
\hline 
311& $  f(p_1)+b(p_2) \to  W^+(\to \nu(p_3)+e^+(p_4))+b(p_5)+f(p_6) $ & LO \\
316& $  f(p_1)+b(p_2) \to  W^-(\to e^-(p_3)+\bar{\nu}(p_4))+b(p_5)+f(p_6) $ & LO \\
\hline 
321& $  f(p_1)+b(p_2) \to  W^+(\to \nu(p_3)+e^+(p_4))+c(p_5)+f(p_6) $ & LO \\
326& $  f(p_1)+b(p_2) \to  W^-(\to e^-(p_3)+\bar{\nu}(p_4))+c(p_5)+f(p_6) $ & LO \\
\hline 
331& $ W^+(\to \nu(p_3)+e^+(p_4))+c(p_5)+f(p_6) ~[ \mbox{c-s interaction} ] $ & LO \\
336& $ W^-(\to e^-(p_3)+\bar{\nu}(p_4))+c(p_5)+f(p_6) ~[ \mbox{c-s interaction} ] $ & LO \\
\hline 
341& $ Z^0(\to e^-(p_3)+e^+(p_4))+b(p_5)+f(p_6) [+f(p_7)]$ & NLO \\
342& $ Z^0(\to e^-(p_3)+e^+(p_4))+b(p_5)+f(p_6) [+b~(p_7)]$ & (REAL) \\
346& $ Z^0(\to e^-(p_3)+e^+(p_4))+b(p_5)+f(p_6)+f(p_7)$ & LO \\
347& $ Z^0(\to e^-(p_3)+e^+(p_4))+b(p_5)+f(p_6)+b~(p_7)$ & LO \\
\hline 
351& $ Z^0(\to e^-(p_3)+e^+(p_4))+c(p_5)+f(p_6) [+f(p_7)]$ & NLO \\
352& $ Z^0(\to e^-(p_3)+e^+(p_4))+c(p_5)+f(p_6) [+c~(p_7)]$ & (REAL) \\
356& $ Z^0(\to e^-(p_3)+e^+(p_4))+c(p_5)+f(p_6)+f(p_7)$ & LO \\
357& $ Z^0(\to e^-(p_3)+e^+(p_4))+c(p_5)+f(p_6)+c~(p_7)$ & LO \\
\hline 
902 --& $  ~[ \mbox{Internal consistency checks} ] $ & -- \\
% 903& $  ~[ \mbox{Check of Volume of 3 particle phase space} ] $ & -- \\
% 904& $  ~[ \mbox{Check of Volume of 4 particle phase space} ] $ & -- \\
% 905& $  ~[ \mbox{Check of Volume of 5 particle phase space} ] $ & -- \\
% 906& $  ~[ \mbox{Check of Volume of 6 particle phase space} ] $ & -- \\
% 908& $  ~[ \mbox{Check of Volume of 8 particle phase space} ] $ & -- \\
% 909& $  ~[ \mbox{Check of Volume of 4 particle massive phase space} ] $ & -- \\
% 910& $  ~[ \mbox{Check of Volume of 3 particle (2 massive) phase space} ] $ & -- \\
\hline
\end{tabular}
\end{center}
\caption{Processes indicated by choice of the variable {\tt nproc}.\label{nproctable}}

\end{table}

\item {\tt part}.
This parameter has 5 possible values, described below:
\begin{itemize}
\item {\tt lord}.
The calculation is performed at leading order only.
\item {\tt virt}.
Virtual (loop) contributions to the next-to-leading order result are
calculated (+counterterms to make them finite), including also the
lowest order contribution.
\item {\tt real}.
In addition to the loop diagrams calculated by {\tt virt}, the full
next-to-leading order results must include contributions from diagrams
involving real gluon emission (-counterterms to make them finite).
Note that only the sum of the {\tt real} and the {\tt virt} contributions
is physical.
\item {\tt tota}.
For simplicity, the {\tt tota} option simply runs the {\tt virt} and
{\tt real} real pieces in series before performing a sum to obtain
the full next-to-leading order result. In this case, the number of
points specified by {\tt ncall1} and {\tt ncall2} is automatically
increased when performing the {\tt real} calculation. In practice,
it may be more efficient to do run the pieces separately by hand, 
(c.f. {\tt ncall} below.
\item {\tt todk}
Processes 161,166,171,176,181,186 only, see sections~\ref{subsec:stop} and
\ref{subsec:wt} below.
\end{itemize}

\item {\tt runstring}.
When {\tt MCFM} is run, it will write output to several files. The
label {\tt runstring} will be appended to the names of these files.

\item {\tt sqrts}. This is the centre-of-mass energy, $\sqrt{s}$ of
the colliding particles, measured in GeV.

\item {\tt ih1}, {\tt ih2}. The identities of the incoming hadrons
may be set with these parameters, allowing simulations for both
$p{\bar p}$ (such as the Tevatron) and $pp$ (such as the LHC). 
Setting {\tt ih1} equal to ${\tt +1}$ corresponds to
a proton, whilst ${\tt -1}$ corresponds to an anti-proton.
Values greater than {\tt 1000d0} represent a nuclear collision,
as described in Section~\ref{sec:nucleus}.

\item {\tt hmass}. For processes involving the Higgs boson, this
parameter should be set equal to the putative value of $M_H$.

\item {\tt scale}. This parameter may be used to adjust the value
of the {\it renormalization} scale. This is the scale
at which $\alpha_S$ is evaluated and will typically be set to
a mass scale appropriate to the process ($M_W$, $M_Z$, $M_t$ for
instance). For processes involving vector bosons, setting this
scale to {\tt -1d0} chooses a scale equal to the average mass of
the bosons involved.

\item {\tt facscale}. This parameter may be used to adjust the value
of the {\it factorization} scale and will typically be set to
a mass scale appropriate to the process ($M_W$, $M_Z$, $M_t$ for
instance). As above, setting it to {\tt -1d0} will choose an
appropriate value for certain processes.

\item {\tt dynamicscale} When set to {\tt .true.} the renormalization and
factorization scales are recalculated for each event. In this case,
the values of {\tt scale} and {\tt facscale} take on a special meaning
that determines the method used to calculate the scale:
  \begin{itemize}
  \item {\tt 1d0} : $\mu^2 = M_V^2+p_T(V)^2$ -- only appropriate for
  processes involving a vector boson $V$
  \item {\tt 2d0} : $\mu^2 = \langle p_T^{jet} \rangle^2$ -- only appropriate for
  processes involving jets
  \item {\tt 3d0} : $\mu^2 = \hat{s}$, the partonic centre of mass energy
  \item {\tt 4d0} : $\mu^2 = H_T^2$, where $H_T$ is the scalar sum of
  the $p_T$'s of all the particles in each event
  \end{itemize}
  Note that care must be used when selecting this option, since not
  all choices are appropriate for each process.

\item {\tt zerowidth}. When set to {\tt .true.} then all vector
bosons are produced on-shell. This is appropriate for calculations
of {\it total} cross-sections (such as when using {\tt removebr} equal
to {\tt .true.}, below). When interested in decay products of the
bosons this should be set to {\tt .false.}.

\item {\tt removebr}. When set to {\tt .true.} the branching ratios are 
removed for unstable particles such as vector bosons or top quarks. See the
process notes in Section~\ref{sec:specific} below for further details.

\item {\tt itmx1}, {\tt itmx2}. The program will perform two runs of
{\tt VEGAS} - once for pre-conditioning and then the final run to
collect the total cross-section and fill histograms. The number of
sweeps for each run is given by {\tt itmx1} (pre-conditioning)
and {\tt itmx2} (final). The default value for both is {\tt 10}.


\item {\tt ncall1}, {\tt ncall2}. For every sweep of {\tt VEGAS},
the number of events generated will be {\tt ncall1} in the
pre-conditioning stage and {\tt ncall2} in the final run. The number
of events required depends upon a number of factors. The error
estimate on a total cross-section will often be reasonable for a
fairly small number of events, whereas accurate histograms will
require a longer run. As the number of particles in the final state
increases, so should the number of calls per sweep. Typically one
might make trial runs with {\tt part} set to {\tt lord} to determine
reasonable values for {\tt ncall1} and {\tt ncall2}. Such values
should also be appropriate for the {\tt virt} piece of
next-to-leading order and should probably be increased by a factor of
$\sim 5$ for the {\tt real} calculation.

\item {\tt ij}. This is the seed for the {\tt VEGAS} integration
and can be altered to give different results for otherwise identical
runs.

\item {\tt dryrun}. The default value of this parameter is
{\tt .false.}. When set to {\tt .true.} the pre-conditioning sweeps
in the {\tt VEGAS} integration are skipped, with the reported
results coming from a single run, (ie {\tt itmx1} iterations of 
{\tt ncall1} points each)
\item {\tt Qflag}. This only has an effect when running a
$W+2$~jets or $Z+2$~jets process. Please see section~\ref{subsec:w2jets}
below.

\item {\tt Gflag}. This only has an effect when running a
$W+2$~jets or $Z+2$~jets process. Please see section~\ref{subsec:w2jets}
below.

\begin{center}
\{blank line\} \\
{\tt [Pdf selection] }
\end{center}

\item {\tt pdlabel}. The choice of parton distribution is made by
inserting the appropriate 7-character code from Table~{\ref{pdlabel}}
here. As mentioned above, this also sets the value of $\alpha_S(M_Z)$.

\item {\tt NGROUP, NSET}. These integers choose the parton distribution
functions to be used when using the PDFLIB package.
\item {\tt LHAPDF group, LHAPDF set}. These choose the parton
distribution functions to be used when using the LHAPDF package --
the group is specified by a character string and the set by an integer.
Please see {\tt http://durpdg.dur.ac.uk/lhapdf/} for further details.
For appropriate PDF sets choosing a value of -1 for the set number ({\tt  LHAPDF set}) 
will perform the calculation of the PDF uncertainties (see also
Section~\ref{sec:histos}, especially the caveat regarding using
non-grid PDF sets).

\begin{center}
\{blank line\} \\
{\tt [Jet definition and event cuts] }
\end{center}

\item {\tt m34min}, {\tt m34max}, {\tt m56min}, {\tt m56max}.
These parameters represent a basic set of cuts that may be applied
to the calculated cross-section. The only events that contribute to
the cross-section will have, for example,
{\tt m34min} $<$ {\tt m34} $<$ {\tt m34max} where {\tt m34} is the
invariant mass of particles 3 and 4 that are specified by {\tt nproc}.
{\tt m34min}~$> 0$ is obligatory for processes which can involve a virtual
photon, such as {\tt nproc=31}.
\item {\tt inclusive}.  This logical parameter chooses whether the
calculated cross-section should be inclusive in the number of jets
found at NLO. An {\em exclusive}
cross-section contains the same number of jets at next-to-leading
order as at leading order. An {\em inclusive} cross-section may
instead contain an extra jet at NLO.

\item {\tt algorithm.} This specifies the jet-finding algorithm that
is used, and can take the values
{\tt ktal} (for the Run II $k_T$-algorithm), {\tt cone} (for
a midpoint cone algorithm), {\tt hqrk} (for a simplified cone
algorithm designed for heavy quark processes) and {\tt none} (to
specify no jet clustering at all). The latter option is only a
sesnible choice when the leading order cross-section is well-defined
without any jet definition: e.g. the single top process,
$q{\bar q^\prime} \to t{\bar b}$, which is finite as
$p_T({\bar b}) \to 0$.

\item {\tt ptmin\_jet, etamin\_jet, etamax\_jet}. These specify the values
of $p_T^{\rm min}$, $|\eta|^{\rm min}$ and $|\eta|^{\rm max}$ for the
jets that are found by the algorithm. 

\item {\tt Rcut\_jet}. If the final state of the chosen process contains
either quarks or gluons then for each event an attempt will be made
to form them into jets. For this it is necessary to define the
jet separation $\Delta R=\sqrt{{\Delta \eta}^2 + {\Delta \phi}^2}$
so that after jet combination, all jet pairs are separated by
$\Delta R >$~{\tt Rcut\_jet}.

\item {\tt makecuts}. If this parameter is set to {\tt .false.} then
no additional cuts are applied to the events and the remaining
parameters in this section are ignored. Otherwise, events will
be rejected according to a set of cuts that is specified below.
Further options may be implemented by editing {\tt src/User/gencuts.f}.

\item {\tt ptmin\_lepton, etamax\_lepton}. These specify the values
of $p_T^{\rm min}$ and $|y|^{\rm max}$ for the hardest lepton produced
in the process.
\item {\tt ptmin\_missing}. Specifies the minimum missing transverse
momentum (coming from neutrinos).
\item {\tt ptmin\_lepton(2nd+), etamax\_lepton(2nd+)}. These specify
the values of $p_T^{\rm min}$ and $|y|^{\rm max}$ for the remaining
leptons in the process. This allows for staggered cuts where, for
instance, only one lepton is required to be hard and central.

\item {\tt R(jet,lept)\_min}. Using the definition of $\Delta R$ above,
requires that all jet-lepton pairs are separated by
$\Delta R >$~{\tt R(jet,lept)\_min}.

\item {\tt R(lept,lept)\_min}. When non-zero, all lepton-lepton pairs
must be separated by $\Delta R >$~{\tt R(lept,lept)\_min}.

\item {\tt Delta\_eta(jet,jet)\_min}. This enforces a pseudo-rapidity
gap between the two hardest jets $j_1$ and $j_2$, so that: \\
$|\eta^{j_1} - \eta^{j_2}| >$~{\tt Delta\_eta(jet,jet)\_min}.

\item {\tt jets\_opphem}. If this parameter is set to {\tt .true.},
then the two hardest jets are required to lie in opposite hemispheres,
$\eta^{j_1} \cdot \eta^{j_2} < 0$.

\item {\tt lepbtwnjets\_scheme}. This integer parameter provides no
additional cuts when it takes the value {\tt 0}. When equal to
{\tt 1} or {\tt 2}, leptons are required to lie between the two
hardest jets. With the ordering $\eta^{j_-} < \eta^{j_+}$ for the
pseudo-rapidities of jets $j_1$ and $j_2$: \\
{\tt lepbtwnjets\_scheme = 1} : 
 $\eta^{j_-} < \eta^{\rm leptons} < \eta^{j_+}$; \\
{\tt lepbtwnjets\_scheme = 2} :
 $\eta^{j_-}+{\tt Rcut\_jet} < \eta^{\rm leptons} < \eta^{j_+}-{\tt Rcut\_jet}$.

\item {\tt ptmin\_bjet,  etamax\_bjet}. If {\tt makecuts} is {\tt .true.}
and a process involving $b$-quarks is being calculated, then these can
be used to specify {\em stricter} values of $p_T^{\rm min}$
and $|\eta|^{\rm max}$ for $b$-jets.

\item {\tt ptminphoton, etamaxphoton}. These specify the values
of $p_T^{\rm min}$ and $|y|^{\rm max}$ for any photons produced.

\item {\tt conephoton, coneptcut}. These constitute a photon isolation
cut which ensures that the amount of hadronic
transverse momentum in a cone around each photon is less than
a specified fraction of the photon's $p_T$.
\begin{displaymath}
\sum_{R < R_0} p_T^{\rm hadronic} < f \times p_T^{photon},
\end{displaymath}
where $R_0$ and $f$ are specified by {\tt cone\_photon} and
{\tt cone\_ptcut} respectively.

\begin{center}
\{blank line\} \\
{\tt [Anomalous couplings of the W and Z] }
\end{center}

\item {\tt Delta\_g1(Z)}. {\it See section~\ref{subsec:diboson}.}
\item {\tt Delta\_K(Z)}. {\it See section~\ref{subsec:diboson}.}
\item {\tt Delta\_K(gamma)}. {\it See section~\ref{subsec:diboson}.}
\item {\tt Lambda(Z)}. {\it See section~\ref{subsec:diboson}.}
\item {\tt Lambda(gamma)}. {\it See section~\ref{subsec:diboson}.}
\item {\tt Form-factor scale, in TeV}. {\it See section~\ref{subsec:diboson}.}

\begin{center}
\{blank line\} \\
{\tt [How to resume/save a run] }
\end{center}

\item {\tt readin}. If {\tt .true.}, the program will read in a
previously saved {\tt VEGAS} grid from the file specified by
{\tt ingridfile.grid}. Note that this, and the following 3 options,
have no effect if {\tt part} is set to {\tt tota} (in this case, grids
are automatically saved and loaded as part of the calculation).

\item {\tt writeout}. If {\tt .true.}, the program will write out
the {\tt VEGAS} grid at the end of the run, to the file specified by
{\tt outgridfile.grid}.

\item {\tt ingridfile}.  {\it See above.}

\item {\tt outgridfile}.  {\it See above.}

\end{itemize}

The final section of the input file contains settings for various technical parameters that should not normally need to be changed. Prior to version 5.5, these were set in {\tt technical.DAT}. For backwards compatibility they may still be specified in that file too, although they will be over-ridden by any settings here.
\begin{center}
\{blank line\} \\
{\tt [Technical parameters that should not normally be changed]}
\end{center}
\begin{itemize}
\item {\tt debug}.  
A logical variable which can be used during a 
debugging phase to mandate special behaviours. 
Passed by common block \\
{\tt common/debug/debug}.
\item {\tt verbose}.  
A logical variable which can be used during a debugging phase to write 
special information. Passed in common block \\
{\tt common/verbose/verbose}.
\item {\tt new\_pspace}.  
A logical variable which can be used during a debugging phase to test altenative versions of the phase space.
Passed in common block {\tt common/new\_pspace/new\_pspace}.
\item {\tt virtonly}.  
A logical variable. The default value for this variable is false.
If virtonly is set to true, during the running of the real part,
the effect of real radiation is neglected, and only the effect of the
integrated dipoles is retained. 
\item {\tt realonly}.  
A logical variable. The default value for this variable is false.
If realonly is set to true, during the running of the real part,
the effect of integrated dipoles is neglected, and only the effect of the
real radiation is retained. 
\item {\tt spira}.  
A logical variable. If {\tt spira} is true, we calculate the 
width of the Higgs boson by interpolating from a table
calculated using the NLO code of M. Spira.
Otherwise the LO value valid for low Higgs masses only is used.
\item {\tt noglue}.  
A logical variable. 
The default value is false. If set to true, no processes
involving initial gluons are included.
\item {\tt ggonly}.  
A logical variable. 
The default value is false. If set to true, 
only the processes
involving initial gluons in both hadrons are included.
\item {\tt gqonly}.  
The default value is false. If set to true, 
only the processes
involving an initial gluon in one hadron and an initial quark
or antiquark in the other hadron (or vice versa) are included.
\item {\tt vanillafiles}.  
A logical variable. The default value is false. If set to true, 
the output files have the generic names 
{\tt mcfm-output.top} and {\tt mcfm-output.dat}. In addition
the path to the parton distribution files is truncated so that
they are expected to be found in the same directory as the executable
{\tt mcfm}.
\item {\tt nmin}
A technical parameter used in alternative phase space generating routines.
\item {\tt nmax}
A technical parameter used in alternative phase space generating routines.
\item {\tt clustering}
This logical parameter determines whether clustering is performed to yield
jets. Only during a debugging phase should this variable be set to false. 
\item {\tt realwt}.  
This is a logical paramter that in general should be set to false.
If set to true, mcfm samples the integral according to the
unsubtracted real emission weight.
\item {\tt colourchoice}.  
If colourchoice=0, all colour structure are included ($W,Z+2$~jets).
If colourchoice=1, only the leading 
colour structure is included ($W,Z+2$~jets).
\item {\tt rtsmin}.  
A minimum value of $\sqrt{s_{12}}$, which ensures that the invariant mass
of the incoming partons can never be less than {\tt rtsmin}.
\item {\tt cutoff}.  
A minimum value of $s_{ij}$, which ensures that the invariant mass squared
of any pair of partons can never be less than {\tt cutoff}.
\item {\tt aii}.  
A double precision variable which can be used to
limit the kinematic range for the subtraction of initial-initial dipoles
as suggested by Trocsanyi and Nagy~\cite{Nagy:2003tz}.   
The value {\tt aii=1d0} corresponds 
to standard Catani-Seymour subtraction.
\item {\tt aif}.  
A double precision variable which can be used to
limit the kinematic range for the subtraction of initial-final dipoles
as suggested by Trocsanyi and Nagy~\cite{Nagy:2003tz}.   
The value {\tt afi=1d0} corresponds 
to standard Catani-Seymour subtraction.
\item {\tt afi}.  
A double precision variable which can be used to
limit the kinematic range for the subtraction of final-initial dipoles
as suggested by Trocsanyi and Nagy~\cite{Nagy:2003tz}.   
The value {\tt afi=1d0} corresponds 
to standard Catani-Seymour subtraction.
\item {\tt aff}. 
A double precision variable which can be used to
limit the kinematic range for the subtraction of final-final dipoles
as suggested by Trocsanyi and Nagy~\cite{Nagy:2003tz}.   
The value {\tt aff=1d0} corresponds 
to standard Catani-Seymour subtraction.
\end{itemize}

\section{Nuclear collisions}
\label{sec:nucleus}

It is possible to specify nuclear collisions by choosing values
of {\tt ih1} and/or {\tt ih2} above {\tt 1000d0}. In that case,
the identity of the nucleus is specified by the atomic number
and mass ($Z$ and $A$ respectively) as follows:
\begin{equation}
{\tt ih} = 1000Z+A.
\end{equation}
For example, to choose an incoming lead beam one would set
{\tt ih1=+82207d0}, corresponding to $Z=82$ and $A=207$.
When running the program, the value of {\tt sqrts} should also be
changed. This must be done by hand and is not automatically taken
care of by the
program. The centre-of-mass energy is decreasd by a factor of
$\sqrt{Z/A}$ for each nuclear beam. 

The nucleon PDF's are calculated by applying the correction
factors of EKS98~\cite{Eskola:1998df} on top of the PDF set that is selected.
This construction simply corrects each parton distribution by
a factor that depends on the value of $(x,\mu)$ in the event.
This parametrization is limited to the region $\mu < 100$~GeV and
any value above that threshold will instead default to $100$~GeV.

Note that the cross-section reported by the program at the end
of the run is given per nucleon per beam. Therefore the
appropriate factors of $A$ should be applied in order to obtain
the total cross section.

\section{Output}
 
In addition to the direct output of the program to {\tt stdout}, after
the final sweep of {\tt VEGAS} the program will output two additional files.
If a working directory was specified in the command line, then these
output files will be written to that directory.

The standard output will detail the iteration-by-iteration best estimate
of the total cross-section, together with the accompanying error estimate.
After all sweeps have been completed, a final summary line will be printed.
In the {\tt npart}~$=$~{\tt tota} case, this last line will actually be the
sum of the two separate real and virtual integrations.
If the {\tt LHAPDF} package is being used and the value of
{\tt LHAPDF set} is equal to {\tt -1},  to indicate a calculation using
PDF uncertainties, then two estimates of the PDF uncertainty are
also supplied in the regular output and also echoed to the file,
{\tt pdferrors.res}. The first estimate corresponds to the
{\tt HEPDATA} method, using the explicit formula given in
Eqn. (43) of Ref.~\cite{Campbell:2006wx}. The second is the so-called
MC prescription, described in detail in Appendix B of
Ref.~\cite{Ball:2008by}~\footnote{
The authors thank M. Ubiali and collaborators for providing their
implementation of the MC method within the MCFM framework.}.


The two other output files are {\tt outputname.dat} and
{\tt outputname.top}, which contain data for various histograms associated
with the calculated process. The first of these is in a raw format 
which may be read in by a plotting package of the user's choosing. The
other file contains the histograms as a {\tt TOPDRAWER} file, as well
as a summary of the options file ({\tt input.DAT}) in the form of
comments at the beginning. The structure
of {\tt outputname} is as follows:
\begin{displaymath}
{\tt procname\_part\_pdlabel\_scale\_facscale\_runstring}
\end{displaymath}
where {\tt procname} is a label assigned by the program corresponding to
the calculated process; the remaining labels are as input by the user
in the file {\tt input.DAT}.

The histograms are  setup in the file {\tt src/User/nplotter.f}. The arguments of this
subroutine are { \tt p,wt,switch}. {\tt p} contains the momenta of all the particles
(i.e. the four momenta of the leptons and jets). The order of the jets is not
necessarily the  order specified in process.DAT.  However in the case that we have a 
$b$-quark or antiquark they are labelled by {\tt bq} and {\tt ba} respectively
in the array jetlabel. {\tt wt} is the weight of the event.  

\subsection{Histograms}
\label{sec:histos}

Extra histograms may be added to the file {\tt src/User/nplotter.f} in
a fairly straightforward manner. Each histogram is filled by making
a call to the routine {\tt bookplot} and updating the histogram
counter {\tt n} by 1. For example:

\begin{verbatim}
  call bookplot(n,tag,'eta3',eta3,wt,wt2,-4d0,4d0,0.1d0,'lin')
  n=n+1
\end{verbatim}
The first two arguments of the call should not be changed. The third
argument is a string describing the plot which will be used for the
title in {\tt TOPDRAWER}. The fourth argument carries the variable to
be plotted, which has been previously calculated. The arguments {\tt
wt} and {\tt wt2} contain information about the phase-space weight and
should not be changed. The
last arguments tell the histogramming routine to use bins of size {\tt
0.1} which run from {\tt -4} to {\tt 4}, and use a linear scale for
the plot. A logarithmic scale may be used by changing the final
argument to {\tt 'log'}.

If the {\tt LHAPDF} package is being used and the value of
{\tt LHAPDF set} is equal to {\tt -1},  to indicate a calculation using
PDF uncertainties, then errors on distributions may also be accumulated.
Note that, due to limitations within the LHAPDF distribution, calculations
using error PDF sets are impractical unless the grid versions of the sets
are used. The grid versions are available in LHAPDF v.3 onwards and may be identified
by the {\tt .LHgrid} extension in the {\tt PDFsets} directory. To use the
grid version, simply pass the PDF set name, including this extension, as
the value of {\tt LHAPDF group} in the input file.
 
To accumulate errors in distributions, add an extra
line to {\tt nplotter.f} after the
call to {\tt bookplot} but before the counter is incremented. For
example, to calculate the PDF uncertainties on the distribution
of {\tt eta3} one would simply add:
\begin{verbatim}
   call ebookplot(n,tag,eta3,wt)
\end{verbatim}
The third argument contains the variable to plot and the other entries
should not be changed. The other parameters for the plot are exactly
those specified on the previous line, in the call to {\tt bookplot}.
Since each PDF error distribution takes up quite a lot of memory
during execution, there is a limit of 4 on the number of distributions
with errors that can be calculated at one time. When calculating
PDF uncertainties on distributions, the program will produce an
additional file which contains the results for each PDF error set
individually. In addition the main file will include the uncertainty
limits on the distribution, which is obtained by choosing the extremal
values of the weights in all PDF sets, on a bin-by-bin basis. Thus
the resulting error limits are not simply described by a single PDF
set.

\subsection{Other output modes}
\label{subsec:otheroutput}
As noted in the description of the input file, there are a number of other
output modes which may be useful in certain situations. In particular, the
ability to output n-tuples can be used to generate a large event record that 
can be subsequently analyzed according to the user's needs. Much of the code
for generating these outputs can be found in {\tt src/User/dswhbook.f}; some
additional work may be required, depending on the process under study.

The simplest alternative output mode is obtained by changing the flag
{\tt dswhisto} to {\tt .true.} . In this way, the {\tt TOPDRAWER} output file is
replaced by the file {\tt outputname.rz} which contains the histograms in {\tt HBOOK}
format.

\subsubsection{Simple n-tuple output}
To obtain the simplest n-tuple output, the flag {\tt creatent} should be set
to {\tt .true.} and the parameter {\tt NTUPLES} in the {\tt Makefile} should be changed
to either {\tt YES} or {\tt FROOT}. When changed to {\tt YES}, each event that enters a histogram
is also recorded as an n-tuple in the file {\tt outputname\_batchno.rz}. The {\tt batchno}
starts at zero and is incremented by one every one millions events. Each event
is a simple row-wise n-tuple consisting of the 4-momenta of each of the final
state particles ($p_x$, $p_y$, $p_z$ and $E$, in that order) followed by 5 numbers
representing the event weight. The first number represents the total event weight
and the others, the contribution from gluon-gluon, quark-gluon (and antiquark-gluon),
quark-quark (and antiquark-antiquark) and quark-antiquark initial states. If
PDF uncertainties are being calculated (using {\tt LHAPDF}) then the total event weight
corresponding to each of the additional PDF sets is also written out at this stage.
Single precision is used, for economy. A simple way to analyze 
these n-tuples is to use the {\tt h2root} command and then perform
manipulations with the ROOT package. Note that these n-tuples contain
no information about either the flavour or the colour of the initial or final
state particles. Summation and averaging over these variables has already been
performed. Furthermore, the `events' are {\it weighted} - so are not events
in the traditional event generator sense.

\subsubsection{n-tuples using FROOT}
Output is similar when using the {\tt FROOT} option. In this case, the program will directly
fill a ROOT n-tuple using the FROOT interface of P. Nadolsky ({\tt nadolsky@pa.msu.edu}, a version
of which is included with MCFM (in the directory {\tt src/User/froot.c}). The structure
of the ntuples is slightly different to that above, with entries:
\begin{itemize}
\item {\tt E\_i}, {\tt pxi}, {\tt pyi}, {\tt pzi} to specify the particle momenta, with {\tt i}
 looping over all members of the final state.
\item {\tt wt\_ALL}, {\tt wt\_gg}, {\tt wt\_gq}, {\tt wt\_qq}, {\tt wt\_qqb} for the total event
 weight and the weights in each of the parton-parton subchannels.
\item {\tt PDFjj}, where {\tt jj} loops over all the PDF uncertainty sets (only written if
 appropriate).
\end{itemize}

\subsubsection{Unweighted events}
{\it Note that this option is both extremely limited and inefficient.}

In order to obtain {\it unweighted} events, one must change the flag
{\tt evtgen} to {\tt .true.} . This option is only available at lowest order
at present. In this mode, the program will first perform a run to obtain the
maximum weight and then perform a simple unweighting procedure against this
number. In this mode the program will also assign identities to the
particles in the initial state, according to the relative parton-parton
luminosities. The routines that handle most of the processing can be found
in {\tt src/User/eventhandler.f}. Much of this code was tailored to diboson
production in Run II of the Tevatron, so further work by the user is almost
certainly required.

\section{Notes on specific processes}
\label{sec:specific}

Note that, as of version 4.0, the version of each process described in the file{\tt process.DAT} includes all appropriate boson decays. This is the 
calculation
that is described when the parameter {\tt removebr} is set to {\tt .false.} ,
as indicated above.

In many cases a more simple calculation can be performed by setting this
parameter to {\tt .true.}, in which case these decays are not performed. In the
notes below we indicate the simpler processes thus obtained. When running in
this mode, the parameter {\tt zerowidth} should be set to {\tt .true.} also,
for consistency. However in certain circumstances, for the sake of comparison,
it may be useful to run with it set to {\tt .false.} .

\subsection{$W$-boson production, processes 1,6}
\label{subsec:wboson}

These processes represent the production of a $W$ boson which subsequently
decays leptonically. The calculation may be performed at NLO.

When {\tt removebr} is true, the $W$ boson does not decay.

\subsection{$W+$~jet production, processes 11,16}
\label{subsec:w1jet}

These processes represent the production of a $W$ boson which subsequently
decays leptonically, in association with a single jet.
The calculation may be performed at NLO.

When {\tt removebr} is true, the $W$ boson does not decay.

\subsection{$W\gamma$ production, processes 12,17}
\label{subsec:wgamma}

These processes represent the production of a $W$ boson which subsequently
decays leptonically, in association with a real photon.
Since this process includes a real photon, the cross section diverges
when the photon is very soft or in the direction of the beam.
Thus in order to produce sensible results, the input file must set
{\tt makecuts} to {\tt .true.} and supply values for both
{\tt ptmin\_photon} and {\tt etamax\_photon}. This will ensure that
the cross section is well-defined.

The calculation may be performed at NLO. Note however that the contribution
in which a jet fragments into a photon is not included, so these processes 
are currently of limited utility.

When {\tt removebr} is true, the $W$ boson does not decay.

\subsection{$W+c$ production, processes 13,18}
\label{subsec:wc}

These processes represent the production of a $W$ boson which
subsequently decays leptonically, in association with a charm
quark. This is produced at leading order by an initial state which
contains a strange quark (or Cabibbo suppressed $d$ quark) and a
gluon.  The effect of the charm quark mass is included throughout the
calculation.  As of version 5.2, the calculation of this process may
be performed at NLO.

When {\tt removebr} is true, the $W$ boson does not decay.

\subsection{$W+c$ production ($m_c=0$), processes 14,19}
\label{subsec:wcmassless}

These processes are identical to {\tt 13} and {\tt 18} except for the fact
that the charm quark mass is neglected. The calculation can currently be
performed at LO only.

\subsection{$W+b{\bar b}$ production, processes 20,25}
\label{subsec:wbb}

These processes represent the production of a $W$ boson which subsequently
decays leptonically, in association with a $b{\bar b}$ pair. The effect of
the bottom quark mass is included throughout the calculation.  
The calculation may be performed at LO only.

When {\tt removebr} is true, the $W$ boson does not decay.

\subsection{$W+b{\bar b}$ production ($m_b=0$), processes 21,26}
\label{subsec:wbbmassless}

These processes are identical to {\tt 20} and {\tt 25} except for the fact
that the bottom quark mass is neglected. This allows the calculation to be
performed up to NLO, with currently calculated virtual matrix elements.

When {\tt removebr} is true, the $W$ boson does not decay.

\subsection{$W+2$~jets production, processes 22,27}
\label{subsec:w2jets}

\begin{center}
[{\it For more details on this calculation, please refer to \break
 hep-ph/0202176 and hep-ph/0308195}]
\end{center}
This process represents the production of a $W$ boson and $2$ jets,
where the $W$ boson decays leptonically. The calculation may be
performed up to NLO, as detailed below.

For these processes (and also for $Z+2$~jet production, {\tt nproc=44})
the next-to-leading order matrix elements are
particularly complex and so they have been divided into two groups.
The division is according to the lowest order diagrams from which they
originate:
\begin{enumerate}
\item Diagrams involving two external quark lines and two external gluons,
the ``{\tt Gflag}'' contribution. The real diagrams in this case thus
involve three external gluons.

\item Diagrams where all four external lines are quarks,
the ``{\tt Qflag}'' contribution. The real diagrams in this case 
involve only one gluon.
\end{enumerate}

By specifying {\tt Gflag} and {\tt Qflag} in {\tt input.DAT} one may
select one of these options at a time. The full result may be obtained
by straightforward addition of the two individual pieces, with no
meaning attached to either piece separately. In the lowest order calculation,
both of these may be set to {\tt .true.} simultaneously - however this is
the only case where this is possible.

When {\tt removebr} is true, the $W$ boson does not decay.

\subsection{$W+3$~jets production, processes 23,28}
\label{subsec:w3jets}

This process represents the production of a $W$ boson and $3$ jets,
where the $W$ boson decays leptonically. The calculation may be
performed at LO only.

When {\tt removebr} is true, the $W$ boson does not decay.

\subsection{$W+b{\bar b}+$~jet production ($m_b=0$), processes 24,29}
\label{subsec:wbbjetmassless}

These processes represent the production of a $W$ boson which subsequently
decays leptonically, in association with a $b{\bar b}$ pair and an
additional jet. The effect of the bottom quark mass is neglected throughout
and the calculation may be performed at LO only.

When {\tt removebr} is true, the $W$ boson does not decay.

\subsection{$Z$-boson production, processes 31--33}
\label{subsec:zboson}

These processes represent the production of a $Z$ boson which subsequently
decays either into electrons ({\tt nproc=31}), neutrinos ({\tt nproc=32})
or bottom quarks ({\tt nproc=33}). Where appropriate, the effect of a virtual
photon is also included. As noted above, in these latter cases {\tt m34min > 0}
is obligatory. The calculation may be performed at NLO,
although the NLO calculation of process {\tt 33} does not include radiation
from the bottom quarks (it is in the initial state only).

When {\tt removebr} is true in process {\tt 31}, the $Z$ boson does not decay.

\subsection{$Z+$~jet production, processes 41--43}
\label{subsec:zjet}

These processes represent the production of a $Z$ boson and a single jet,
where the $Z$ subsequently
decays either into electrons ({\tt nproc=41}), neutrinos ({\tt nproc=42})
or bottom quarks ({\tt nproc=43}). Where appropriate, the effect of a virtual
photon is also included. The calculation may be performed at NLO,
although the NLO calculation of process {\tt 43} does not include radiation
from the bottom quarks.

When {\tt removebr} is true in process {\tt 41}, the $Z$ boson does not decay.

\subsection{$Z+2$~jets production, process 44}
\label{subsec:z2jets}

\begin{center}
[{\it For more details on this calculation, please refer to \break
 hep-ph/0202176 and hep-ph/0308195}]
\end{center}

This process represents the production of a $Z$ boson and $2$ jets,
including also the effect of a virtual photon. The $Z/\gamma^*$ decays
to an $e^+ e^-$ pair.  The calculation may be performed up to NLO -
please see the earlier Section~\ref{subsec:w2jets} for more details,
especially the discussion {\tt Qflag} and {\tt Gflag}.

When {\tt removebr} is true, the $Z$ boson does not decay.

\subsection{$Z+3$~jets production, process 45}
\label{subsec:z3jets}

This process represents the production of a $Z$ boson and $3$ jets,
including also the effect of a virtual photon. The $Z/\gamma^*$ decays
to an $e^+ e^-$ pair.  The calculation may be performed at LO only.

When {\tt removebr} is true, the $Z$ boson does not decay.

\subsection{$Z\gamma$ production, processes 48,49}
\label{subsec:zgamma}

These processes represent the production of a $Z$ boson (or virtual photon)
in association with a real photon. The $Z/\gamma^*$ subsequently decays into 
either an $e^+ e^-$ pair ({\tt nproc=48}) or neutrinos ({\tt nproc=49}).
Since these processes include a real photon, the cross section diverges
when the photon is very soft or in the direction of the beam.
Thus in order to produce sensible results, the input file must set
{\tt makecuts} to {\tt .true.} and supply values for both
{\tt ptmin\_photon} and {\tt etamax\_photon}. This will ensure that
the cross section is well-defined.

The calculation may be performed at NLO. However since the contribution
of a jet fragmenting into a photon is not included, 
these processes are currently of limited utility.


When {\tt removebr} is true in process {\tt 48}, the $Z$ boson does not decay.

\subsection{$Z+b{\bar b}$ production, process 50}
\label{subsec:zbb}

These processes represent the production of a $Z$ boson (or virtual photon)
which subsequently decays leptonically, in association
with a $b{\bar b}$ pair. The effect of
the bottom quark mass is included throughout the calculation.  
The calculation may be performed at LO only.

When {\tt removebr} is true, the $Z$ boson does not decay.

\subsection{$Z+b{\bar b}$ production ($m_b=0$), processes 51--53}
\label{subsec:zbbmassless}

Process {\tt 51} is identical to {\tt 50} except for the fact
that the bottom quark mass is neglected. This allows the calculation to be
performed up to NLO. The other processes account for the decays into
neutrinos ({\tt nproc=52}) and bottom quarks ({\tt nproc=53}). Note that
the NLO calculation of process {\tt 53} does not currently 
include radiation from the
bottom quarks produced in the decay.

When {\tt removebr} is true in process {\tt 51}, the $Z$ boson does not decay.

\subsection{$Z+b{\bar b}+$~jet production ($m_b=0$), process 54}
\label{subsec:zbbjetmassless}

This process represents the production of a $Z$ boson (and virtual photon)
which subsequently decays leptonically, in association
with a $b{\bar b}$ pair and an additional jet.
The effect of the bottom quark mass is neglected throughout
and the calculation may be performed at LO only.

When {\tt removebr} is true, the $Z$ boson does not decay.

\subsection{$Z+c{\bar c}$ production ($m_c=0$), process 56}
\label{subsec:zccmassless}

Process {\tt 56} is the equivalent of {\tt 51}, with the bottom quarks
replaced by charm. Although the charm mass is neglected, the calculation
contains diagrams with two gluons in the initial state and a
$Z$ coupling to the quark line -- hence the dependence upon the quark
flavour.

When {\tt removebr} is true in process {\tt 56}, the $Z$ boson does not decay.

\subsection{Di-boson production, processes 61--89}
\label{subsec:diboson}

\begin{center}
[{\it For more details on this calculation, please refer to hep-ph/9905386}]
\end{center}

These processes represent the production of a diboson pair $V_1 V_2$,
where $V_1$ and $V_2$ may be either a $W$ or $Z/\gamma^*$. 
All the processes in this section may be calculated at NLO, with the exception
of {\tt nproc=64}. There are various
possibilities for the subsequent decay of the bosons, as specified in the
sections below.

\subsubsection{$WW$ production, processes 61-64}

For $WW$ production, both $W$'s can decay leptonically ({\tt nproc=61}) or one
may decay hadronically ({\tt nproc=62} for $W^-$ and {\tt nproc=63} for $W^+$).
Process {\tt 64} implements the matrix elements for the leptonic decay of
both $W$'s but where no polarization information is retained. It is included
for the sake of comparison with other calculations. Note that, in processes
{\tt 62} and {\tt 63}, the NLO corrections do not include radiation from the
hadronic decays of the $W$.

When {\tt removebr} is true in processes {\tt 61} and {\tt 64},
the $W$ bosons do not decay.

\subsubsection{$WW$+jet production, process 66}

This process is only implemented for the leptonic decay modes of both $W$
bosons and is currently limited to LO accuracy only. When {\tt removebr} is true,
the $W$ bosons do not decay.

\subsubsection{$WZ$ production, processes 71--73, 76--78}

For $WZ$ production, the $W$ is chosen to decay leptonically. The $Z$ (or
virtual photon, when appropriate) may decay into electrons
({\tt nproc=71},{\tt 76}), neutrinos ({\tt nproc=72},{\tt 77}) or a
pair of bottom quarks ({\tt nproc=73},{\tt 78}).
Note that, in processes
{\tt 73} and {\tt 78}, the NLO corrections do not include radiation from the
bottom quarks that are produced by the $Z$ decay.

When {\tt removebr} is true in processes {\tt 71} and {\tt 76},
neither the $W$ or the $Z$ boson decays.

\subsubsection{$ZZ$ production, processes 81--84, 86--89}

For $ZZ$ production, there are two sets of processes corresponding to the
inclusion of a virtual photon when appropriate ({\tt nproc=81}--{\tt 84})
and the case where it is neglected ({\tt nproc=86}--{\tt 89}).
Thus {\tt nproc=86}--{\tt 89} are really for diagnostic purposes only.

The $Z$'s can either both decay leptonically ({\tt nproc=81},{\tt 86}),
one can decay leptonically while the other decays into neutrinos
({\tt nproc=82},{\tt 87}) or bottom quarks ({\tt nproc=83},{\tt 88}), or
one decays into neutrinos and the other into a bottom quark pair
({\tt nproc=84},{\tt 89}). Note that, in processes
{\tt 83}--{\tt 84} and {\tt 88}--{\tt 89}, the NLO corrections do not include
radiation from the bottom quarks that are produced by the $Z$ decay.

When {\tt removebr} is true in processes {\tt 81} and {\tt 86},
neither of the $Z$ bosons decays.

\subsubsection{$ZZ$+jet production, process 85}

This process is only implemented for the case when one $Z$ boson decays to
electrons and the other to neutrinos (i.e. the companion of {\tt nproc=82}).
It may only be calculated at LO. When {\tt removebr} is true, the $Z$ bosons
do not decay.

\subsubsection{Anomalous couplings}

As of version 3.0, it is possible to specify anomalous trilinear
couplings for the $W^+W^-Z$ and $W^+W^-\gamma$ vertices that are
relevant for $WW$ and $WZ$ production. To run in this mode, one
must set {\tt zerowidth} equal to {\tt .true.}
and modify the appropriate lines for the couplings in {\tt input.DAT},
(see below).

The anomalous couplings appear in the Lagrangian,
${\cal L} = {\cal L}_{SM} + {\cal L}_{anom}$ as follows
(where ${\cal L}_{SM}$ represents the usual Standard Model Lagrangian):
\begin{eqnarray}
{\cal L}_{anom} & = & i g_{WWZ} \Biggl[
 \Delta g_1^Z \left( W^*_{\mu\nu}W^\mu Z^\nu - W_{\mu\nu}W^{*\mu} Z^\nu \right)
+\Delta\kappa^Z W^*_\mu W_\nu Z^{\mu\nu} \nonumber \\
 & &+
 \frac{\lambda^Z}{M_W^2} W^*_{\rho\mu} W^\mu_\nu Z^{\nu\rho} \Biggr]
+i g_{WW\gamma} \Biggl[ 
 \Delta\kappa^\gamma W^*_\mu W_\nu \gamma^{\mu\nu}
+\frac{\lambda^\gamma}{M_W^2} W^*_{\rho\mu} W^\mu_\nu\gamma^{\nu\rho}
 \Biggr], \nonumber
\end{eqnarray}
where $X_{\mu\nu} \equiv \partial_\mu X_{\nu} - \partial_\nu X_{\mu}$
and the overall coupling factors are $g_{WWZ}=-e$,
$g_{WW\gamma}=-e\cot\theta_w$.
This is the most general Lagrangian that conserves $C$ and $P$
separately and electromagnetic gauge invariance requires that there
is no equivalent of the $\Delta g_1^Z$ term for the photon coupling.

In order to avoid a violation of unitarity, these couplings are included
in {\tt MCFM} only after suppression by dipole form factors,
\begin{displaymath}
\Delta g_1^Z \rightarrow \frac{\Delta g_1^Z}{(1+\hat{s}/\Lambda^2)^2}, \qquad
\Delta \kappa^{Z/\gamma} \rightarrow
 \frac{\Delta \kappa_1^{Z/\gamma}}{(1+\hat{s}/\Lambda^2)^2}, \qquad
\lambda^{Z/\gamma} \rightarrow
 \frac{\Delta \lambda^{Z/\gamma}}{(1+\hat{s}/\Lambda^2)^2},
\end{displaymath}
where $\hat{s}$ is the vector boson pair invariant mass and $\Lambda$
is an additional parameter giving the scale of new physics, which should
be in the TeV range.
These form factors should be produced by the new physics associated with the
anomalous couplings and this choice is somewhat arbitrary.

The file {\tt input.DAT} contains the values of the $6$ parameters
which specify the anomalous couplings:
\begin{verbatim}
    0.0d0           [Delta_g1(Z)]
    0.0d0           [Delta_K(Z)]
    0.0d0           [Delta_K(gamma)]
    0.0d0           [Lambda(Z)]
    0.0d0           [Lambda(gamma)]
    2.0d0           [Form-factor scale, in TeV]
\end{verbatim}
with the lines representing $\Delta g_1^Z$, $\Delta \kappa^Z$,
$\Delta \kappa^\gamma$, $\lambda^Z$, $\lambda^\gamma$ and
$\Lambda$~[TeV] respectively. By setting the first 5 parameters to zero,
as above, one recovers the Standard Model result.

\subsection{$WH$ production, processes 91, 92, 96, 97}
\label{subsec:wh}

These processes represent the production of a $W$ boson which subsequently
decays leptonically, in association with a Standard Model Higgs boson that
decays into a bottom quark pair ({\tt nproc=91, 96}) or a pair of $W$ bosons
({\tt nproc=92, 97}).  Note that in the latter case, below the $W$ pair threshold
one of the $W$ bosons is virtual and therefore one must set {\tt zerowidth=.false.}.
The calculation may be performed at NLO, although radiation from the
bottom quarks in the decay is not included.

When {\tt removebr} is true, neither the $W$ boson nor the Higgs decays.

\subsection{$ZH$ production, processes 101--103, 106--108}
\label{subsec:zh}

These processes represent the production of a $Z$ boson (or virtual photon)
in association with a Standard Model Higgs boson that
decays into a bottom quark pair ({\tt nproc=101-103}) or a pair of $W$ bosons
({\tt nproc=106-108}). The $Z/\gamma^*$ subsequently decays into 
either an $e^+ e^-$ pair ({\tt nproc=101, 106}), neutrinos ({\tt nproc=102, 107})
or a bottom quark pair ({\tt nproc=103, 108}).
The calculation may be performed at NLO, although radiation from the
bottom quarks in the decay of the Higgs (or the $Z$, for processes
{\tt 103, 108}) is not included.

When {\tt removebr} is true in processes {\tt 101, 106}, neither the $Z$ boson
nor the Higgs decays.

\subsection{Higgs production, processes 111--116}
\label{subsec:h}

These processes represent the production of a Standard Model Higgs
boson that decays either into a bottom quark
pair ({\tt nproc=111}), a pair of tau's ({\tt nproc=112}), a $W^+W^-$ pair
that further decays leptonically ({\tt nproc=113}) or a $ZZ$ pair.
For the case of a $ZZ$ decay,
the subsequent decays can either be into a pair of muons and a pair of electrons
({\tt nproc=114)}, a pair of electrons and neutrinos ({\tt nproc=115}) or
a pair of muons and a pair of bottom quarks ({\tt nproc=116}).

The calculation is performed in the
limit of infinite top quark mass, so that at LO the relevant diagram
is the coupling of two gluons to the Higgs via a top quark loop.
The calculation may be performed at NLO, although radiation from the
bottom quarks in the decay of processes {\tt 111} and {\tt 116} is not included.

When {\tt removebr} is true in processes {\tt 111}--{\tt 114},
the Higgs boson does not decay.

\subsection{$H+b$ production, processes 141--143}
\label{subsec:Hb}

\begin{center}
[{\it For more details on this calculation, please refer to hep-ph/0204093}]
\end{center}

These processes represent the production of a Standard Model Higgs
boson that decays into a pair of bottom quarks,
in association with a further bottom quark. The initial state at lowest order
is a bottom quark and a gluon.
The calculation may be performed at NLO, although radiation from the
bottom quarks in the Higgs decay is not included.

For this process, the matrix elements are divided up into a number of
different sub-processes, so the user must sum over these after performing
more runs than usual. At lowest order one can proceed as normal, using
{\tt nproc=141}. For a NLO calculation, the sequence of runs is as follows:
\begin{itemize}
\item Run {\tt nproc=141} with {\tt part=virt} and {\tt part=real} (or, both
at the same time using {\tt part=tota});
\item Run {\tt nproc=142} with {\tt part=real}.
\end{itemize}
The sum of these yields the cross-section with one identified $b$-quark in
the final state. To calculate the contribution with two $b$-quarks in the
final state, one should use {\tt nproc=143} with {\tt part=real}.

When {\tt removebr} is true, the Higgs boson does not decay.

\subsection{$t\bar{t}$ production with decay, processes 151 and 152}

These processes describe lowest order $t \bar{t}$ production 
including two leptonic decays $t \to b l \nu$ (process {\tt 151}) and 
one leptonic and one hadronic decay ({\tt 152}).
We use the matrix elements of ref.~\cite{Kleiss:1988xr}.

%In the case of process 152, there is no radiation off the decay products
%of the $W$ which decays hadronically.

When {\tt removebr} is true in process {\tt 151},
the top quarks do not decay.

\subsection{$t\bar{t}$ production with decay and a gluon, process 156}
This process describes lowest order $t \bar{t}+g$ production 
including two leptonic decays $t \to b l \nu$. Zerowidth must be set to true for 
this case.

When {\tt removebr} is true, the top quarks do not decay.

\subsection{$Q\overline{Q}$ production, processes 157--159}
These processes calculate the production of heavy quarks
({\tt 157} for top, {\tt 158} for bottom and {\tt 159} for charm) up to NLO 
using the matrix elements of ref.~\cite{Nason:1987xz}. No decays
are included.

\subsection{$t{\bar t}+$~jet production, process 160}
This process calculates the production of top quarks and a single jet
at LO, without any decay of the top quarks.

\subsection{Single top production, processes 161--177}
\label{subsec:stop}

\begin{center}
[{\it For more details on this calculation, please refer to hep-ph/0408158}]
\end{center}

These processes represent single top production and may be calculated up to
NLO as described below.

Single top production is divided as usual into $s$-channel 
(processes {\tt 171-177}) and $t$-channel ({\tt 161-167})
diagrams. Each channel includes separately the production of a top
and anti-top quark, which is necessary when calculating rates at the LHC.
Below we illustrate the different use of these processes by considering
$t$-channel top production ({\tt 161,162}), although the procedure is the same
for anti-top production ({\tt 166,167}) and the corresponding $s$-channel
processes ({\tt 171,172}) and ({\tt 176,177}).

To calculate cross-sections that do not include any decay of the (anti-)top
quark, one should use process {\tt 161}
(or, correspondingly, {\tt 166}, {\tt 171} and {\tt 176}) with {\tt removebr}
true. The procedure is exactly the same
as for any other process.

When one wishes to calculate observables related to the decay of the top
quark, {\tt removebr} should be false.
The LO calculation proceeds as normal. At NLO, there are two options:
\begin{itemize}
\item {\tt part=virt, real} or {\tt tota} : final state radiation is included
in the production stage only
\item {\tt part = todk} : radiation is included in the decay of the top
quark also and the final result corresponds to the sum of real and virtual
diagrams. This process can only be performed at NLO with 
{\tt zerowidth = .true}. This should be set automatically.
Note that these runs automatically perform an extra integration, so
will take a little longer.
\end{itemize}

The contribution from radiation in the decay may be calculated separately using
process {\tt 162}. This process number can be used with {\tt part=virt,real}
only. To ensure consistency, it is far simpler to use {\tt 161}
and this is the recommended approach.

\subsection{$Wt$ production, processes 180--186}
\label{subsec:wt}

\begin{center}
[{\it For more details on this calculation, please refer to hep-ph/0506289}]
\end{center}

These processes represent the production of a $W$ boson that decays leptonically
in association with a top quark. The lowest order diagram involves a gluon and
a bottom quark from the PDF, with the $b$-quark radiating a $W$ boson and
becoming a top quark. The calculation can be performed up to NLO.

Processes {\tt 180} and {\tt 185} produce a top quark that does not decay,
whilst in processes {\tt 181} and {\tt 186} the top quark decays leptonically.
Consistency with
the simpler processes ({\tt 180,185}) can be demonstrated by running process
{\tt 181,186} with {\tt removebr} set to true.

At next-to-leading order, the calculation includes contributions from diagrams
with two gluons in the initial state, $gg \rightarrow Wtb$. The $p_T$ of the
additional $b$ quark is vetoed according to the value of the parameter
{\tt ptmin\_bjet} which is specified in the input file. The contribution from
these diagrams when the $p_T$ of the $b$ quark is above {\tt ptmin\_bjet}
is zero. The values of this parameter and the factorization scale ({\tt facscale})
set in the input file should be chosen carefully. Appropriate values for both
(in the range $30$-$100$~GeV) are discussed in the associated paper.

When one wishes to calculate observables related to the decay of the top
quark, {\tt removebr} should be false.
The LO calculation proceeds as normal. At NLO, there are two options:
\begin{itemize}
\item {\tt part=virt, real} or {\tt tota} : final state radiation is included
in the production stage only
\item {\tt part = todk} : radiation is included in the decay of the top
quark also and the final result corresponds to the sum of real and virtual
diagrams. This process can only be performed at NLO with 
{\tt zerowidth = .true}. This should be set automatically.
Note that these runs automatically perform an extra integration, so
will take a little longer.
\end{itemize}

The contribution from radiation in the decay may be calculated separately using
processes {\tt 182,187}. These process numbers can be used with {\tt part=virt,real}
only. To ensure consistency, it is far simpler to use {\tt 181,186}
and this is the recommended approach.

\subsection{$Ht{\bar t}$ production, processes 190 and 191}
\label{subsec:htt}

These processes represent the production of a Higgs boson in association
with a pair of top quarks. The calculation can be performed at LO only.

For process {\tt 190}, neither the top quarks nor the Higgs boson
decays. In process {\tt 191}, both the top quarks decay leptonically
and the Higgs boson decays into a pair of bottom quarks. 
Consistency with
the simpler process ({\tt 190}) can be demonstrated by running process
{\tt 191} with {\tt removebr} set to true.

\subsection{$Zt{\bar t}$ production, processes 196 and 197}
\label{subsec:ztt}

These processes represent the production of a $Z$ boson in association
with a pair of top quarks which both decay leptonically.
In process {\tt 196} the $Z$ boson decays into an electron pair, whilst
in {\tt 197} the decay is into a bottom quark pair.
The calculations can be performed at LO only.

When {\tt removebr} is true in process {\tt 196}, the $Z$ boson does not decay.

\subsection{$H+$~jet production, processes 201--208}
\label{subsec:hjet}

These processes represent the production of a Higgs boson in association
with a single jet, with the subsequent decay of the Higgs to either
a pair of bottom quarks (processes {\tt 201,203,206}) 
or to a pair of tau's ({\tt 202,204,207}).
or to a pair of W's which decay lepronically ({\tt 208}).

The Higgs boson couples to a pair of gluons via a loop of heavy fermions
which, in the Standard Model, is accounted for almost entirely by including
the effect of the top quark alone. For processes {\tt 201,202,206,207}, the
matrix elements include the full dependence on the top quark mass.
The calculation can only be performed at LO. 
However, the Higgs boson can either be the Standard Model one
(processes {\tt 201,202}) or a pseudoscalar ({\tt 206,207}).

For processes {\tt 203} and {\tt 204}, the calculation is performed in the
limit of infinite top quark mass, so that NLO results can be obtained.
The virtual matrix elements have been implemented from
refs~\cite{Ravindran:2002dc} and~\cite{Schmidt:1997wr}.
Phenomenological results have previously been 
given in refs.~\cite{deFlorian:1999zd},\cite{Ravindran:2002dc} 
and \cite{Glosser:2002gm}.
Note that the effect of radiation from the bottom quarks in process {\tt 203}
is not included.

When {\tt removebr} is true in processes {\tt 201}, {\tt 203}, {\tt 206}
and {\tt 208}, the Higgs boson does not decay.

\subsection{Higgs production via WBF, processes 211--217}
\label{subsec:wbf}

\begin{center}
[{\it For more details on this calculation, please refer to hep-ph/0403194}]
\end{center}

These processes provide predictions for the production of a Higgs boson in
association with two jets via weak-boson fusion (WBF). The Higgs boson
subsequently decays to either a pair of bottom quarks
(processes {\tt 211, 216}), to a pair of tau's ({\tt 212, 217}) or to
a pair of $W$ bosons ({\tt 213}).

Calculations can be performed up to NLO for processes {\tt 211}, {\tt 212} and {\tt 213}.
In addition to this, processes {\tt 216} and {\tt 217} provide the lowest
order calculation of the WBF reaction which radiates an additional jet. 

When {\tt removebr} is true, the Higgs boson does not decay.

\subsection{$\tau^+\tau^-$ production, process 221}
\label{subsec:tautau}

This process provides predictions for the production of a tau lepton
pair, with subsequent leptonic decays. The calculation is available at LO
only. The relevant matrix elements are adapted from the ones in
ref.~\cite{Kleiss:1988xr}.

When {\tt removebr} is true, the tau leptons do not decay.

\subsection{$Z+Q$ production, processes 261--267}
\label{subsec:ZQ}

\begin{center}
[{\it For more details on this calculation, please refer to hep-ph/0312024}]
\end{center}

These processes represent the production of a $Z$
boson that decays into a pair of electrons,
in association with a heavy quark, $Q$.

For processes {\tt 261}, {\tt 262}, {\tt 266} and {\tt 267} the initial
state at lowest order is the heavy quark and a gluon and 
the calculation may be performed at NLO.
As for $H+b$ production, the matrix elements are divided into two
sub-processes at NLO. Thus the user must sum over these after performing
more runs than usual. At lowest order one can proceed as normal, using
{\tt nproc=261} (for $Z+b$) or {\tt nproc=262} (for $Z+c$).
For a NLO calculation, the sequence of runs is as follows:
\begin{itemize}
\item Run {\tt nproc=261} (or {\tt 262}) with {\tt part=virt} and
{\tt part=real} (or, both at the same time using {\tt part=tota});
\item Run {\tt nproc=266} (or {\tt 267}) with {\tt part=real}.
\end{itemize}
The sum of these yields the cross-section with one identified heavy quark in
the final state when {\tt inclusive} is set to {\tt .false.} . To calculate the
rate for at least one heavy quark, {\tt inclusive} should be {\tt .true.}.

For processes {\tt 263} and
{\tt 264}, the calculation uses the matrix elements for the production
of a $Z$ and a heavy quark pair and demands that one of the heavy quarks
is not observed. It may either lie outside the range of $p_T$ and $\eta$
required for a jet, or both quarks may be contained in the same jet.
Due to the extra complexity (the calculation must retain the full
dependence on the heavy quark mass), this can only be computed at LO.

When {\tt removebr} is true, the $Z$ boson does not decay.

\subsection{$H + 2$~jet production, processes 271, 272}

These processes represent the production of a Standard Model Higgs boson
in association with two jets. The Higgs boson
subsequently decays to either a bottom quark pair ({\tt nproc=271}), 
or a pair of tau's ({\tt nproc=272}).

The matrix elements are included in the infinite top mass limit
using an effective Lagrangian approach. These calculations can be
performed at LO only.

When {\tt removebr} is true, the Higgs boson does not decay.

\subsection{$H + 3$~jet production, processes 273, 274}

These processes represent the production of a Standard Model Higgs boson
in association with three jets. The Higgs boson
subsequently decays to either a bottom quark pair ({\tt nproc=273}), 
or a pair of tau's ({\tt nproc=274}).

The matrix elements are included in the infinite top mass limit
using an effective Lagrangian approach. These calculations can be
performed at LO only.

When {\tt removebr} is true, the Higgs boson does not decay.

\subsection{$W+Q+$~jet production processes 311--326}
\label{subsec:wQj}

These processes represent the production of a $W$
boson that decays leptonically,
in association with a heavy quark, $Q$ and an additional light jet. In
processes {\tt 311} and {\tt 316} $Q$ is a bottom quark, whilst
processes {\tt 321} and {\tt 326} involve a charm quark.
In these processes the quark $Q$ occurs as parton PDF in the initial state. 
The initial state in these processes consists of a light quark and a heavy 
quark, with the light quark radiating the $W$ boson. These calculations may
be performed at LO only.

When {\tt removebr} is true, the $W$ boson does not decay.

\subsection{$W+c+$~jet production, processes 331, 336}
\label{subsec:wcj}

These processes represent the production of a $W$
boson that decays leptonically,
in association with a charm quark and an additional light jet. 

In contrast to processes {\tt 321} and {\tt 326} described above, the initial
state in this case consists of two light quarks, one of which is a
strange quark which radiates the $W$ boson. The calculation may
be performed at LO only.

When {\tt removebr} is true, the $W$ boson does not decay.

\subsection{$Z+Q+$jet production, processes 341--357}
\label{subsec:ZQj}

\begin{center}
[{\it For more details on this calculation, please refer to hep-ph/0510362}]
\end{center}

These processes represent the production of a $Z$
boson that decays into a pair of electrons,
in association with a heavy quark, $Q$ and an untagged jet.

For processes {\tt 341} and {\tt 351} the initial
state at lowest order is the heavy quark and a gluon and 
the calculation may be performed at NLO.
Thus in these processes the quark $Q$ occurs as 
parton PDF in the initial state. 
As for $H+b$ and $Z+Q$ production, the matrix elements are divided into two
sub-processes at NLO. Thus the user must sum over these after performing
more runs than usual. At lowest order one can proceed as normal, using
{\tt nproc=341} (for $Zbj$) or {\tt nproc=351} (for $Zcj$).
For a NLO calculation, the sequence of runs is as follows:
\begin{itemize}
\item Run {\tt nproc=341} (or {\tt 351}) with {\tt part=virt} and
{\tt part=real} (or, both at the same time using {\tt part=tota});
\item Run {\tt nproc=342} (or {\tt 352}) with {\tt part=real}.
\end{itemize}
The sum of these yields the cross-section with one identified heavy quark and one
untagged jet in
the final state when {\tt inclusive} is set to {\tt .false.} . To calculate the
rate for at least one heavy quark and one jet (the remaining jet may be a heavy quark,
or untagged), {\tt inclusive} should be {\tt .true.}.

Processes {\tt 346,347} and {\tt 356,357} are the lowest order processes that enter
the above calculation in the real contribution. They can be computed only at LO.

When {\tt removebr} is true, the $Z$ boson does not decay.

\begin{thebibliography}{99}
%
%\cite{Georgi:1991ci}
\bibitem{Georgi:1991ci}
H.~Georgi,
%``Effective field theory and electroweak radiative corrections,''
Nucl.\ Phys.\ B {\bf 363}, 301 (1991).
%%CITATION = NUPHA,B363,301;%%

%\cite{Maltoni:2002qb}
\bibitem{Maltoni:2002qb}
F.~Maltoni and T.~Stelzer,
%``MadEvent: Automatic event generation with MadGraph,''
JHEP {\bf 0302}, 027 (2003)
[arXiv:hep-ph/0208156].
%%CITATION = HEP-PH 0208156;%%

\bibitem{Alpgen}
Michelangelo L. Mangano, Mauro Moretti, Fulvio Piccinini, Roberto Pittau and Antonello Polosa, {\it http://mlm.home.cern.ch/mlm/alpgen/}

\bibitem{Lusifer}
Stefan Dittmaier and Markus Roth,\\
{\it http://wwwth.mppmu.mpg.de/members/roth/Lusifer/lusifer.html}

%\cite{Eskola:1998df}
\bibitem{Eskola:1998df}
  K.~J.~Eskola, V.~J.~Kolhinen and C.~A.~Salgado,
  %``The scale dependent nuclear effects in parton distributions for  practical
  %applications,''
  Eur.\ Phys.\ J.\ C {\bf 9}, 61 (1999)
  [arXiv:hep-ph/9807297].
  %%CITATION = HEP-PH 9807297;%%
  
%\cite{Campbell:2006wx}
\bibitem{Campbell:2006wx}
  J.~M.~Campbell, J.~W.~Huston and W.~J.~Stirling,
  %``Hard interactions of quarks and gluons: A primer for LHC physics,''
  Rept.\ Prog.\ Phys.\  {\bf 70}, 89 (2007)
  [arXiv:hep-ph/0611148].
  %%CITATION = RPPHA,70,89;%%

%\cite{Ball:2008by}
\bibitem{Ball:2008by}
  R.~D.~Ball {\it et al.}  [NNPDF Collaboration],
  %``A determination of parton distributions with faithful uncertainty
  %estimation,''
  Nucl.\ Phys.\  B {\bf 809}, 1 (2009)
  [arXiv:0808.1231 [hep-ph]].
  %%CITATION = NUPHA,B809,1;%%

%\cite{Kleiss:1988xr}
\bibitem{Kleiss:1988xr}
R.~Kleiss and W.~J.~Stirling,
%``Top Quark Production At Hadron Colliders: Some Useful Formulae,''
Z.\ Phys.\ C {\bf 40}, 419 (1988).
%%CITATION = ZEPYA,C40,419;%%

%\cite{Nason:1987xz}
\bibitem{Nason:1987xz}
P.~Nason, S.~Dawson and R.~K.~Ellis,
%``The Total Cross-Section For The Production Of Heavy Quarks In Hadronic
%Collisions,''
Nucl.\ Phys.\ B {\bf 303}, 607 (1988).
%%CITATION = NUPHA,B303,607;%%

\bibitem{Ravindran:2002dc}
V.~Ravindran, J.~Smith and W.~L.~Van Neerven,
%``Next-to-leading order QCD corrections to differential distributions of  Higgs boson production in hadron hadron collisions,''
Nucl.\ Phys.\ B {\bf 634}, 247 (2002)
[arXiv:hep-ph/0201114].
%%CITATION = HEP-PH 0201114;%%

%\cite{Schmidt:1997wr}
\bibitem{Schmidt:1997wr}
C.~R.~Schmidt,
%``H $\to$ g g g (g q anti-q) at two loops in the large-M(t) limit,''
Phys.\ Lett.\ B {\bf 413}, 391 (1997)
[arXiv:hep-ph/9707448].
%%CITATION = HEP-PH 9707448;%%

%\cite{deFlorian:1999zd}
\bibitem{deFlorian:1999zd}
D.~de Florian, M.~Grazzini and Z.~Kunszt,
%``Higgs production with large transverse momentum in hadronic collisions  at next-to-leading order,''
Phys.\ Rev.\ Lett.\  {\bf 82}, 5209 (1999)
[arXiv:hep-ph/9902483].
%%CITATION = HEP-PH 9902483;%%

%\cite{Glosser:2002gm}
\bibitem{Glosser:2002gm}
C.~J.~Glosser and C.~R.~Schmidt,
%``Next-to-leading corrections to the Higgs boson transverse momentum  spectrum in gluon fusion,''
JHEP {\bf 0212}, 016 (2002)
[arXiv:hep-ph/0209248].
%%CITATION = HEP-PH 0209248;%%

%\cite{Nagy:2003tz}
\bibitem{Nagy:2003tz}
  Z.~Nagy,
  %``Next-to-leading order calculation of three jet observables in hadron hadron
  %collision,''
  Phys.\ Rev.\  D {\bf 68}, 094002 (2003)
  [arXiv:hep-ph/0307268].
  %%CITATION = PHRVA,D68,094002;%%

\end{thebibliography}


\end{document}
