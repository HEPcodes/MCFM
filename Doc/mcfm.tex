\documentclass[12pt]{article}

\begin{document}
\def\GeV{\mbox{GeV}}

\thispagestyle{empty}
\vspace*{3cm}
\begin{center}
{\Huge MCFM v3.4.1} \\
\vspace*{0.5cm}
\Large{A Monte Carlo for FeMtobarn} \\
\Large{processes at Hadron Colliders} \\
\vspace*{2cm}
{\huge Users Guide} \\
\vspace*{4cm}
{\it Authors:} \\
\vspace*{0.2cm}
J. M. Campbell ({\tt johnmc@hep.anl.gov}) \\
R. K. Ellis ({\tt ellis@fnal.gov}) \\
\vspace*{2.5cm}
{\it \small Updated: 29th May 2003}
\end{center}

\newpage

\section{Installation}

The tar'ed, gzip'ed and uu-encoded package may be downloaded from
the {\tt MCFM} home-page at {\tt http://mcfm.fnal.gov}.
After extracting, the source can be initialized by running the
{\tt Install} command and then compiled with {\tt make}. The
{\tt Install} script may be edited prior to running to include
the locations of the CERNLIB and LHAPDF libraries, if desired.
The code has been developed and tested under Redhat Linux, please report
any compilation problems under other operating systems to the authors.

The directory structure of the installation is as follows:
\begin{itemize}
\item {\tt Doc}. The source for this document.
\item {\tt Bin}. The directory containing the executable {\tt mcfm},
and various essential files -- notably the options file {\tt input.DAT}.
\item {\tt Bin/Pdfdata}. The directory containing the PDF data-files.
\item {\tt obj}. The object files produced by the compiler. 
\item {\tt src}. The Fortran source files in various subdirectories.
\end{itemize}

\section{Input parameters}

{\tt MCFM} now allows the user to choose between a number of schemes
for defining the electroweak couplings. These choices are summarized
in Table~\ref{ewscheme}. The scheme is selected by modifying the
value of {\tt ewscheme} in {\tt src/User/mdata.f}, which also contains
the values of all input parameters (see also Table~\ref{default}).

\begin{table}
\begin{center}
\begin{tabular}{|c|c|c|c|c|c|c|} \hline
 Paramter & Name & Input Value
 & \multicolumn{4}{c|}{Output Value determined by \tt ewscheme} \\
\cline{4-7}
& ({\tt \_inp}) & & {\tt -1} & {\tt 0} & {\tt 1} & {\tt 2} \\ \hline
$G_F$            & {\tt Gf}      & 1.16639$\times$10$^{-5}$ 
 & input & calculated & input & input \\
$\alpha(M_Z)$    & {\tt aemmz}   & 1/128.89                 
 & input & input & calculated & input \\
$\sin^2 \theta_w$& {\tt xw}      & 0.2312               
 & calculated & input & calculated & input \\
$M_W$            & {\tt wmass}   & 80.419 GeV                
 & input & calculated & input & calculated \\
$M_Z$            & {\tt zmass}   & 91.188 GeV               
 & input & input & input & calculated \\
$m_t$            & {\tt mt}      & 175 GeV                  
 & calculated & input & input & input \\
\hline
\end{tabular}
\caption{Different options for the scheme used to fix the electroweak
parameters of the Standard Model and the corresponding default input
values.}
\label{ewscheme}
\end{center}
\end{table}

For the default scheme ({\tt ewscheme=-1}), we use the effective field
theory approach, which is valid for scales below the top mass. In this
approach there are 4 independent parameters (which we choose to be
$G_F$, $\alpha(M_Z)$, $M_W$ and $M_Z$). For further details,
see Georgi, {\it Nucl. Phys.} {\bf B363} (1991) 301. 

For the alternative schemes ({\tt ewscheme=0,1,2}) the top mass is simply
an additional input parameter and there are 3 other independent
parameters from the remaining 5. The variable {\tt ewscheme} then performs
exactly the same role as {\tt idef} in MadEvent~\cite{Maltoni:2002qb}.
{\tt ewscheme=0} is the old MadEvent default and {\tt ewscheme=1} is the
new MadEvent default, which is also the same as that used in Alpgen and
LUSIFER. 

\begin{table}
\begin{center}
\begin{tabular}{|c|c|c|} \hline
Parameter & Fortran name & Default value \\ 
\hline
$m_\tau$         & {\tt mtau}      & 1.777 GeV            \\
$m^2_\tau(Q=100\GeV)$& {\tt mtausq}  & 3.1602 GeV$^2$     \\
$m^2_c(Q=100\GeV)$   & {\tt mcsq}    & 0.4  GeV$^2$       \\
$m^2_b(Q=100\GeV)$   & {\tt mbsq}    & 10.7 GeV$^2$       \\
$\Gamma_\tau$    & {\tt tauwidth}& 2.269$\times$10$^{-12}$~GeV \\
$\Gamma_W$       & {\tt wwidth}  & 2.06 GeV               \\
$\Gamma_Z$       & {\tt zwidth}  & 2.49 GeV               \\
$V_{ud}$         & {\tt Vud}     & 0.975                  \\
$V_{us}$         & {\tt Vus}     & 0.22220486             \\
$V_{ub}$         & {\tt Vub}     & 0.                     \\
$V_{cd}$         & {\tt Vcd}     & 0.22220486             \\
$V_{cs}$         & {\tt Vcs}     & 0.975                  \\
$V_{cb}$         & {\tt Vcb}     & 0.                     \\
\hline
\end{tabular}
\caption{Default values for the remaining parameters in {\tt MCFM}.}
\label{default} 
\end{center}
\end{table}

\subsection{Parton distributions}
The value of $\alpha_S(M_Z)$ is not adjustable; it is hardwired with the
parton distribution. In addition, the parton distribution also specifies
the number of loops that should be used in the running of $\alpha_S$.
The default mode of operation is to choose from a
collection of modern parton distribution functions that are included with
MCFM.  The distributions, together with their associated $\alpha_S(M_Z)$
values, are given in the table below. For the older distributions, where the
coupling was specified by $\Lambda$ this requires some calculation/guesswork.
\begin{table}[h]
\begin{center}
\begin{tabular}{|c|c||c|c|}
\hline
{\tt mrs02nl}   & 0.1197       & {\tt mrs02nn}  & 0.1154  \\
{\tt mrs0119}   & 0.119        & {\tt mrs0117}  & 0.117   \\
{\tt mrs0121}   & 0.121        & {\tt mrs01\_j} & 0.121   \\
{\tt mrs99\_1}  & 0.1175       & {\tt mrs99\_2} & 0.1175  \\
{\tt mrs99\_3}  & 0.1175       & {\tt mrs99\_4} & 0.1125  \\    
{\tt mrs99\_5}  & 0.1225       & {\tt mrs99\_6} & 0.1178  \\    
{\tt mrs99\_7}  & 0.1171       & {\tt mrs99\_8} & 0.1175  \\    
{\tt mrs99\_9}  & 0.1175       & {\tt mrs9910}  & 0.1175  \\    
{\tt mrs9911}   & 0.1175       & {\tt mrs9912}  & 0.1175  \\    
{\tt mrs98z1}   &  0.1175      & {\tt cteq5hq}  &  0.118  \\
{\tt mrs98z2}   &  0.1175      & {\tt cteq5f3}  &  0.106  \\
{\tt mrs98z3}   &  0.1175      & {\tt cteq5f4}  &  0.112  \\
{\tt mrs98z4}   &  0.1125      & {\tt mtungb1}  &  0.109  \\
{\tt mrs98z5}   &  0.1225      & {\tt cteq4\_m} &  0.116  \\
{\tt mrs96r1}   &  0.113       & {\tt cteq4\_d} &  0.116  \\
{\tt mrs96r2}   &  0.120       & {\tt cteq4\_l} &  0.132  \\
{\tt mrs96r3}   &  0.113       & {\tt cteq4a1}  &  0.110  \\
{\tt mrs96r4}   &  0.120       & {\tt cteq4a2}  &  0.113  \\
{\tt hmrs90e}   &  0.098382675 & {\tt cteq4a3}  &  0.116  \\
{\tt hmrs90b}   &  0.107961191 & {\tt cteq4a4}  &  0.119  \\
{\tt cteq5\_m}  &  0.118       & {\tt cteq4a5}  &  0.122  \\
{\tt cteq5\_d}  &  0.118       & {\tt cteq4hj}  &  0.116  \\
{\tt cteq5\_l}  &  0.127       & {\tt cteq4lq}  &  0.114  \\
{\tt cteq5l1}   &  0.127       & {\tt cteq5hj}  &  0.118  \\
{\tt cteq6\_m}  &  0.118       & {\tt cteq6\_d} &  0.118  \\
{\tt cteq6\_l}  &  0.118       & {\tt cteq6l1}  &  0.130  \\
\hline
\end{tabular}
\end{center}
\caption{Available pdf sets and their corresponding values of
$\alpha_S(M_Z)$.}
\label{pdlabel}
\end{table}

By editing the {\tt Makefile}, it is straightforward to switch to
either the {\tt PDFLIB} or the {\tt LHAPDF} parton distribution
function implementations.

To use {\tt PDFLIB}, one must first set the variable {\tt CERNLIB}
in the makefile to point to the directory that contains
{\tt libpdflib804.a} and then modify {\tt PDFROUTINES} to
take the value {\tt PDFLIB}. The parameters to choose the
pdf set are then specified in {\tt Bin/input.DAT}.

To use {\tt LHAPDF}, one must first set the variable {\tt LHAPDFLIB}
in the makefile to point to the directory that contains
{\tt libLHAPDF.a} and then modify {\tt PDFROUTINES} to
take the value {\tt LHAPDF}. The parameters to choose the
pdf set are then provided in {\tt Bin/input.DAT} - 
the name of the group and the integer specifying 
the set.
{\tt MCFM} expects to find the sets in a sub-directory of {\tt Bin} called
{\tt PDFsets}, as in the {\tt LHAPDF} distribution. It is easiest to
simply create a symbolic link appropriately.

One may always return to the built-in distributions by setting
{\tt PDFROUTINES} to take the value {\tt NATIVE}.

\section{Runtime options}

{\tt mcfm} execution is performed in the {\tt Bin/} directory,
with syntax:
\begin{center}
{\tt mcfm [}{\it myfile}{\tt .DAT] [}{\it mydir}{\tt ]}
\end{center}
If no command line options are given, then {\tt mcfm} will default
to using the file {\tt input.DAT} in the current directory for
choosing options\footnote{Note that this is very different from
previous versions of {\tt MCFM}. All auxiliary input files from v3.2 and
earlier have now been incorporated into a single file.}.
The different possibilities are summarized in Table~\ref{clopts}.
\begin{table}
\begin{center}
\begin{tabular}{l|cl}
Command executed && Location of input file \\
\hline
{\tt mcfm}                      && {\tt input.DAT} \\
{\tt mcfm myfile.DAT}           && {\tt myfile.DAT} \\
{\tt mcfm mydir}                && {\tt mydir/input.DAT} \\
{\tt mcfm mydir myfile.DAT}     && {\tt mydir/myfile.DAT} \\
\end{tabular}
\end{center}
\caption{Summary of command line options for running {\tt mcfm}.}
\label{clopts}
\end{table}
In addition, if a working directory {\it mydir} is specified then
output files will also be produced in this directory. By using these
options one may, for instance, keep all input and output files for
different processes in separate directories.

Each parameter in the input file is specified by a line such as
\begin{displaymath}
{\tt value} \hspace{3cm} {\tt [parameter]}
\end{displaymath}
and we will give a description of all the parameters below, together with
valid and/or sensible inputs for ${\tt value}$. Groups of parameters
are separated by a blank line and a description of that section, for
readability.

\begin{itemize}
\item {\tt file version number}. This should match the version number
that is printed when {\tt mcfm} is executed.

\begin{center}
\{blank line\} \\
{\tt [Flags to specify the mode in which MCFM is run] }
\end{center}

\item {\tt evtgen}. The default for this, and the following three
parameters, is {\tt .false.} and this corresponds to the usual mode
of operation. It is possible to generate n-tuples instead of histograms,
as well as unweighted events, for some processes. Please contact the
authors for further information.
\item {\tt creatent}. {\it See above.}
\item {\tt skipnt}. {\it See above.}
\item {\tt dswhisto}. {\it See above.}

\begin{center}
\{blank line\} \\
{\tt [General options to specify the process and execution] }
\end{center}

\item {\tt nproc}.
The process to be studied is given by
choosing a process number, according to the following table.
$f(p_i)$ denotes a generic partonic jet.

\begin{table}
\vspace*{-1.5cm}
\begin{center}
\begin{tabular}{|l|l|l|}
\hline
{\tt nproc} & $f(p_1)+f(p_2) \to \ldots $ & Order \\
\hline
0   & $ W^+ ~\mbox{for total cross-section--No $W$-BR} $& NLO \\ 
1   & $ W^+(\to \nu(p_3)+e^+(p_4)) $& NLO \\ 
5   & $ W^- ~\mbox{for total cross-section--No $W$-BR} $& NLO \\ 
6   & $ W^-(\to e^-(p_3)+\bar{\nu}(p_4)) $& NLO \\ 
\hline
10  & $ W^+ + f(p_5)(No $W$ BR)$& NLO \\ 
11  & $ W^+(\to \nu(p_3)+e^+(p_4)) + f(p_5)$& NLO \\ 
16  & $ W^-(\to e^-(p_3)+\bar{\nu}(p_4)) + f(p_5) $& NLO \\ 
18  & $ W^+(\to \nu(p_3)+e^+(p_4)) + \gamma(p_5)$& NLO \\ 
19  & $ W^-(\to e^-(p_3)+\bar{\nu}(p_4)) + \gamma(p_5) $& NLO \\ 
\hline
20  & $ W^+(\to \nu(p_3)+e^+(p_4)) + b(p_5)+\bar{b}(p_6)~\mbox{(massive $b$)}$& LO \\
21  & $ W^+(\to \nu(p_3)+e^+(p_4)) + b(p_5)+\bar{b}(p_6)$& NLO \\
22  & $ W^+(\to \nu(p_3)+e^+(p_4))+f(p_5)+f(p_6)$& NLO \\
23  & $ W^+(\to \nu(p_3)+e^+(p_4))+f(p_5)+f(p_6)+f(p_7)$& LO \\
24  & $ W^+(\to \nu(p_3)+e^+(p_4))+b(p_5)+b~(p_6)+f(p_7)$& LO \\
25  & $ W^-(\to e^-(p_3)+\bar{\nu}(p_4)) + b(p_5)+\bar{b}(p_6)~\mbox{(massive $b$)}$& LO \\
26  & $ W^-(\to e^-(p_3)+\bar{\nu}(p_4))+b(p_5)+\bar{b}(p_6)$& NLO \\
27  & $ W^-(\to e^-(p_3)+\bar{\nu}(p_4))+f(p_5)+f(p_6)$& NLO \\ 
28  & $ W^-(\to e^-(p_3)+\bar{\nu}(p_4))+f(p_5)+f(p_6)+f(p_7)$& LO \\ 
29  & $ W^-(\to e^-(p_3)+\bar{\nu}(p_4))+b(p_5)+b~(p_6)+f(p_7)$& LO \\
\hline
30  & $ Z^0 ~\mbox{for total cross-section--No $Z$-BR} $& NLO \\ 
31  & $ Z^0(\to e^-(p_3)+e^+(p_4)) $& NLO \\ 
32  & $ Z_0(\to 3*(\nu(p_3)+\bar{\nu}(p_4))) $& NLO \\ 
33  & $ Z^0(\to b(p_3)+\bar{b}(p_4)) $& NLO \\ 
\hline
40  & $ Z^0~\mbox{(no BR)} +f(p_5)$& NLO \\
41  & $ Z^0(\to e^-(p_3)+e^+(p_4)) +f(p_5)$& NLO \\
42  & $ Z_0(\to 3*(\nu(p_3)+\bar{\nu}(p_4)))-(\mbox{sum over 3} \nu)+f(p_5)$& NLO \\ 
43  & $ Z^0(\to b(p_3)+\bar{b}(p_4))+f(p_5)$& NLO \\ 
44  & $ Z^0(\to e^-(p_3)+e^+(p_4))+f(p_5)+f(p_6)$& NLO \\ 
45  & $ Z^0(\to e^-(p_3)+e^+(p_4))+f(p_5)+f(p_6)+f(p_7)$& LO \\ 
48  & $ Z^0(\to e^-(p_3)+e^+(p_4)) +\gamma(p_5)$& NLO \\
49  & $ Z_0(\to 3*(\nu(p_3)+\bar{\nu}(p_4)))-(\mbox{sum over 3} \nu)+\gamma(p_5)$& NLO \\ 
\hline
50  & $ Z^0(\to e^-(p_3)+e^+(p_4))+b(p_5)+\bar{b}(p_6)~\mbox{(massive $b$)}$& LO \\ 
51  & $ Z^0(\to e^-(p_3)+e^+(p_4))+b(p_5)+\bar{b}(p_6)$& NLO \\
52  & $ Z_0(\to 3*(\nu(p_3)+\bar{\nu}(p_4)))+b(p_5)+\bar{b}(p_6)$& NLO \\
53  & $ Z^0(\to b(p_3)+\bar{b}(p_4))+b(p_5)+\bar{b}(p_6)$& NLO \\
\hline
60  & $ W^+ + W^- ~\mbox{for total cross-section--No $W$-BRs} $& NLO \\ 
61  & $ W^+(\to \nu(p_3)+e^+(p_4)) +W^-(\to e^-(p_5)+\bar{\nu}(p_6)) $& NLO \\ 
62  & $ W^+(\to \nu(p_3)+e^+(p_4)) +W^-(\to q(p_5)+{\bar q}(p_6)) $& NLO \\ 
63  & $ W^+(\to q(p_3)+{\bar q}(p_4)) +W^-(\to e^-(p_5)+\bar{\nu}(p_6)) $& NLO \\ 
\hline
\end{tabular}
\end{center}
\end{table}
\begin{table}
\vspace*{-1.5cm}
\begin{center}
\begin{tabular}{|l|l|l|}
\hline
{\tt nproc} & $f(p_1)+f(p_2) \to \ldots $& Order \\
\hline
70  & $ Z^0 + W^+ ~\mbox{for total cross-section--No $W$or $Z$-BRs} $& NLO \\ 
71  & $ Z^0(\to e^-(p_5)+e^+(p_6))+W^+(\to \nu(p_3)+\mu^+(p_4)) $& NLO \\ 
72  & $ Z^0(\to 3*(\nu(p_5)+\bar{\nu}(p_6)))+W^+(\to \nu(p_3)+e^+(p_4)) $& NLO \\ 
73  & $ Z^0(\to b(p_5)+\bar{b}(p_6))+W^+(\to \nu(p_3)+e^+(p_4)) $& NLO \\ 
\hline
75  & $ Z^0 + W^- ~\mbox{for total cross-section--No $W$or $Z$-BRs} $& NLO \\ 
76  & $ Z^0(\to e^-(p_5)+e^+(p_6))+W^-(\to \mu^-(p_3)+\bar{\nu}(p_4)) $& NLO \\ 
77  & $ Z^0(\to \nu(p_5)+\bar{\nu}(p_6))+W^-(\to e^-(p_3)+\bar{\nu}(p_4)) $& NLO \\ 
78  & $ Z^0(\to b(p_5)+\bar{b}(p_6))+W^-(\to e^-(p_3)+\bar{\nu}(p_4)) $& NLO \\ 
\hline
80  & $ Z^0 + Z^0 ~\mbox{for total cross-section--No $Z$-BRs} $& NLO \\ 
81  & $ Z^0(\to e^-(p_5)+e^+(p_6))+Z^0(\to \mu^-(p_3)+\mu^+(p_4)) $& NLO \\ 
82  & $ Z^0(\to e^-(p_5)+e^+(p_6))+Z^0(\to 3*(\nu(p_3)+\bar{\nu}(p_4))) $& NLO \\ 
83  & $ Z^0(\to e^-(p_5)+e^+(p_6))+Z^0(\to b(p_3)+\bar{b}(p_4)) $& NLO \\ 
84  & $ Z^0(\to b(p_5)+\bar{b}(p_6))+Z^0(\to 3*(\nu(p_3)+\bar{\nu}(p_4))) $& NLO \\ 
\hline
85  & $ Z^0 + Z^0 ~\mbox{for total cross-section} ~(\mathrm{no}~\gamma^*) $& NLO \\ 
86  & $ Z^0(\to e^-(p_5)+e^+(p_6))+Z^0(\to \mu^-(p_3)+\mu^+(p_4)) ~(\mathrm{no}~\gamma^*) $& NLO \\ 
87  & $ Z^0(\to e^-(p_5)+e^+(p_6))+Z^0(\to 3*(\nu(p_3)+\bar{\nu}(p_4))) ~(\mathrm{no}~\gamma^*) $& NLO \\ 
88  & $ Z^0(\to e^-(p_5)+e^+(p_6))+Z^0(\to b(p_3)+\bar{b}(p_4)) ~(\mathrm{no}~\gamma^*) $& NLO \\ 
89  & $ Z^0(\to
b(p_5)+\bar{b}(p_6))+Z^0(\to 3*(\nu(p_3)+\bar{\nu}(p_4))) ~(\mathrm{no}~\gamma^*) $& NLO \\ 
\hline
90  & $ H + W^+ ~\mbox{for total cross-section--No $W$ or $H$-BRs} $& NLO \\ 
91  & $ H(\to b(p_5)+\bar{b}(p_6)) + W^+(\to \nu(p_3)+e^+(p_4)) $& NLO \\ 
95  & $ H + W^- ~\mbox{for total cross-section--No $W$ or $H$-BRs} $& NLO \\ 
96  & $ H(\to b(p_5)+\bar{b}(p_6)) + W^-(\to e^-(p_3)+\bar{\nu}(p_4)) $& NLO \\ 
\hline
100 & $ H + Z^0 ~\mbox{for total cross-section--No $Z$ or $H$-BRs} $& NLO \\ 
101 & $ H(\to b(p_5)+\bar{b}(p_6)) + Z^0(\to e^-(p_3)+e^+(p_4)) $& NLO \\ 
102 & $ H(\to b(p_5)+\bar{b}(p_6)) + Z^0(\to 3*(\nu(p_3)+\bar{\nu}(p_4))) $& NLO \\ 
103 & $ H(\to b(p_5)+\bar{b}(p_6)) + Z^0(\to b(p_3)+\bar{b}(p_4)) $& NLO \\ 
\hline
110 & $ H~\mbox{for total cross-section} $& NLO \\ 
111 & $ H(\to b(p3) + {\bar b}(p4)) $& NLO \\ 
112 & $ H(\to W^- W^+) ~\mbox{for total cross-section--No $H$ or $W$-BRs} $& NLO \\ 
113 & $ H(\to W^-(e^-(p_5)+\bar{\nu}(p_6))+W^+(\nu(p_3)+e^+(p_4))) $& NLO \\ 
\hline
121 & $ H(\to Z^0(e^-(p_5)+e^+(p_6))+Z^0(\mu^-(p_3)+\mu^+(p_4)) $& NLO \\ 
122 & $ H(\to Z^0(e^-(p_5)+e^+(p_6))+Z^0(3*(\nu(p_3)+\bar{\nu}(p_4))) $& NLO \\ 
123 & $ H(\to Z^0(e^-(p_5)+e^+(p_6))+Z^0(b(p_3)+b~(p_4)) $& NLO \\ 
\hline
131 & $ H-->t(-->nu(p_3)+e^+(p_4)+b(p_5))+b~(p_6))+e^-(p_7)+nu~(p_8) $& NLO \\ 
\hline
140 & $ H ~\mbox{(no BR)} + b(p_5)$& NLO \\
141 & $ H ~\mbox{(no BR)} + b(p_5) +g(p_6)$& NLO \\
142 & $ H ~\mbox{(no BR)} + {\bar b}(p_5) +b(p_6)$& NLO \\
143 & $ H ~\mbox{(no BR)} + b(p_5) + {\bar q}(p_6)$& NLO \\
144 & $ H ~\mbox{(no BR)} + b(p_5) + {\bar b}(p_6)$& NLO \\
145 & $ H ~\mbox{(no BR)} + b(p_5) + {\bar b}(p_6) ~\mbox{(both observed)}$& LO \\
\hline
\end{tabular}
\end{center}
\end{table}


\begin{table}
\begin{center}
\hspace*{-1.5cm}
\begin{tabular}{|l|l|l|}
\hline
{\tt nproc} & $f(p_1)+f(p_2) \to \ldots $& Order \\
\hline
150 & $ t \bar{t} ~\mbox{for total cross-section} $& LO \\ 
156 & $ t(\to\nu(p_3)+e^+(p_4)+b(p_5))+{\bar t}(\to
\bar{b}(p_6))+W^-(\to {\bar \nu}(p_7)+e^-(p_8))$& LO \\ 
\hline
161 & $ t(\to\nu(p_3)+e^+(p_4)+b(p_5))+\bar{b}(p_6))+q(p_7)~(\mathrm{missing}) $& LO \\ 
\hline
171 & $ t(\to \nu(p_3)+e^+(p_4)+b(p_5))+\bar{b}(p_6)) $& LO \\ 
\hline
180 & $ \tau^- \tau^+ ~\mbox{for total cross-section-Drell-Yan} $& NLO \\ 
181 & $ \tau^-(\to \nu(p_3)+e^+(p_4)+\nu(p_5))+\tau^+(\to \bar{\nu}(p_6)+e^-(p_8)+\bar{\nu}(p_9)) $& LO \\  
\hline
190 & $ t(p_3) + {\bar t}(p_4) + H(p_5) $& LO \\ 
191 & $ t(p_3+p_4+p_5) + {\bar t}(p_6+p_7+p_8) + H(p_9+p_{10}) $& LO \\ 
192 & $ t(\to \nu(p_3)+e^+(p_4)+b(p_5)) +
{\bar t}(\to {\bar b}(p_6)+{\bar \nu}(p_7)+e^-(p_8)) + H(p_9+p_{10}) $& LO \\ 
193 & $ t(p_3+p_4+p_5)+{\bar t}(p_6+p_7+p_8)+Z(p_9+p_{10})$& LO \\
194 & $ t(p_3+p_4+p_5)+{\bar t}(p_6+p_7+p_8)+Z(p_9+p_{10})$&LO \\
195 & $ t(\to \nu(p_3)+e^+(p_4)+b(p_5))+{\bar t}(\to {\bar \nu}(p_7)+e^-(p_8)+{\bar b}(p_6))+Z(e(p_9)+e^-(p_{10}))$&LO \\
196 & $ t(\to \nu(p_3)+e^+(p_4)+b(p_5))+{\bar t}(\to {\bar \nu}(p_7)+e^-(p_8)+{\bar b}(p_6))+Z(b(p_9)+{\bar b}(p_{10}))$&LO \\
\hline
200 & $ H(p_3+p_4) + f(p_5) ~\mbox{for total cross-section-No $H$ BR} $& LO \\ 
201 & $ H(\to \tau(p_3)+{\bar \tau}(p_4)) + f(p_5)$& LO \\ 
202 & $ H(\to \tau^- (\to \nu_\tau(p_3)+e^-(p_4)+{\bar \nu_e}(p_5))
+\tau^+(\to {\bar \nu_\tau}(p_6)+\nu_e(p_7)+e^+(p_8))) + f(p_9)$& LO \\ 
202 & $ H(\to \tau^- (\to \nu_\tau(p_3)+e^-(p_4)+{\bar \nu_e}(p_5))
+\tau^+(\to {\bar \nu_\tau}(p_6)+\nu_e(p_7)+e^+(p_8))) + f(p_9)$ & NLO\\ 
205 &$  A(p_3+p_4) + f(p_5) ~\mbox{(for total cross-section-No $A$ BR)}$&   NLO \\
206 &$  A (\to \tau(p_3) + {\bar \tau}(p_4)) + f(p_5)$&   NLO \\
211 &$  H(p_3+p_4)+f(p_5)+f(p_6) ~\mbox{ (VV-fusion + interference)}$& LO \\
212 &$  H(p_3+p_4)+f(p_5)+f(p_6)+f(p_7) ~\mbox{ (VV-fusion + interference)}$& LO \\
213 &$  H(p_3+p_4)+f(p_5)+f(p_6) ~\mbox{ (WW-fusion for total Xsect)}$& NLO \\
214 &$  H(p_3+p_4)+f(p_5)+f(p_6)+f(p_7) ~\mbox{ (WW-fusion for total Xsect)}$& NLO \\
215 &$  H(p_3+p_4)+f(p_5)+f(p_6) ~\mbox{ (ZZ-fusion for total Xsect)}$& NLO \\   
216 &$  H(p_3+p_4)+f(p_5)+f(p_6)+f(p_7) ~\mbox{ (ZZ-fusion for total Xsect)}$& NLO \\
217 &$  H(p_3+p_4)+f(p_5)+f(p_6) ~\mbox{ (VV-fusion for total Xsect)}$& NLO \\
218 &$  H(p_3+p_4)+f(p_5)+f(p_6)+f(p_7) ~\mbox{ (VV-fusion for total Xsect)}$& NLO \\
\hline
30$n$ &  Check of volume of $n$-particle phase space& --  \\
\hline
\end{tabular}
\end{center}
\end{table}


\item {\tt part}.
This parameter has 4 possible values, described below:
\begin{itemize}
\item {\tt lord}.
The calculation is performed at leading order only.
\item {\tt virt}.
Virtual (loop) contributions to the next-to-leading order result are
calculated (+counterterms to make them finite), including also the
lowest order contribution.
\item {\tt real}.
In addition to the loop diagrams calculated by {\tt virt}, the full
next-to-leading order results must include contributions from diagrams
involving real gluon emission (-counterterms to make them finite).
\item {\tt tota}.
For simplicity, the {\tt tota} option simply runs the {\tt virt} and
{\tt real} real pieces in series before performing a sum to obtain
the full next-to-leading order result. In this case, the number of
points specified by {\tt ncall1} and {\tt ncall2} is automatically
increased when performing the {\tt real} calculation. In practice,
it may be more efficient to do run the pieces separately by hand as
described below.
\end{itemize}

\item {\tt runstring}.
When {\tt MCFM} is run, it will write output to several files. The
label {\tt runstring} will be appended to the names of these files.

\item {\tt sqrts}. This is the centre-of-mass energy, $\sqrt{s}$ of
the colliding particles, measured in GeV.

\item {\tt ih1}, {\tt ih2}. The identities of the incoming hadrons
may be set with these parameters, allowing simulations for both
$p{\bar p}$ (such as the Tevatron) and $pp$ (such as the LHC)
simulations. Setting {\tt ih1} equal to ${\tt +1}$ corresponds to
a proton, whilst ${\tt -1}$ corresponds to an anti-proton.

\item {\tt hmass}. For processes involving the Higgs boson, this
parameter should be set equal to the putative value of $M_H$.

\item {\tt scale}. This parameter may be used to adjust the value
of the renormalization and factorization scale. This is the scale
at which $\alpha_S$ is evaluated and will typically be set to
a mass scale appropriate to the process ($M_W$, $M_Z$, $M_t$ for
instance).
For the case where particles $3$ and $4$ are produced by the decay of
a vector boson $V$, there are two special scale choices. Setting
{\tt scale} equal to {\tt -1d0} selects a scale equal to the mass of
the boson, $M_V$. An input of {\tt scale} equal to {\tt -2d0}
generates the scale for each event individually, with a value
given by $\sqrt{M_V^2+p_T(V)^2}$.

\item {\tt zerowidth}. When set to {\tt .true.} then all vector
bosons are produced on-shell. This is appropriate for calculations
of {\it total} cross-sections such as {\tt nproc} $=$ {\tt 60},
{\tt 70} and {\tt 80}. When interested in decay products of the
bosons (such as {\tt nproc} $=$ {\tt 61}), this should be set
to {\tt .false.}.

\item {\tt itmx1}, {\tt itmx2}. The program will perform two runs of
{\tt VEGAS} - once for prec-conditioning and then the final run to
collect the total cross-section and fill histograms. The number of
sweeps for each run is given by {\tt itmx1} (pre-conditioning)
and {\tt itmx2} (final). The recommended values are the default,
{\tt 10}.

\item {\tt ncall1}, {\tt ncall2}. For every sweep of {\tt VEGAS},
the number of events generated will be {\tt ncall1} in the
pre-conditioning stage and {\tt ncall2} in the final run. The number
of events required depends upon a number of factors. The error
estimate on a total cross-section will often be reasonable for a
fairly small number of events, whereas accurate histograms will
require a longer run. As the number of particles in the final state
increases, so should the number of calls per sweep. Typically one
might make trial runs with {\tt part} set to {\tt lord} to determine
reasonable values for {\tt ncall1} and {\tt ncall2}. Such values
should also be appropriate for the {\tt virt} piece of
next-to-leading order and should probably be increased by a factor of
$\sim 5$ for the {\tt real} calculation.

\item {\tt ij}. This is the seed for the {\tt VEGAS} integration
and can be altered to give different results for otherwise identical
runs.

\item {\tt dryrun}. The default value of this parameter is
{\tt .false.}. When set to {\tt .true.} the pre-conditioning sweeps
in the {\tt VEGAS} integration are skipped, with the reported
results coming from a single run.
\item {\tt Qflag}. This only has an effect when running a
$W+2$~jets or $Z+2$~jets process. Please see section~\ref{subsec:v2j}
below.

\item {\tt Gflag}. This only has an effect when running a
$W+2$~jets or $Z+2$~jets process. Please see section~\ref{subsec:v2j}
below.

\begin{center}
\{blank line\} \\
{\tt [Pdf selection] }
\end{center}

\item {\tt pdlabel}. The choice of parton distribution is made by
inserting the appropriate 7-character code from Table~{\ref{pdlabel}}
here. As mentioned above, this also sets the value of $\alpha_S(M_Z)$.

\item {\tt NGROUP, NSET}. These integers choose the parton distribution
functions to be used when using the PDFLIB package.
\item {\tt LHAPDF group, LHAPDF set}. These choose the parton
distribution functions to be used when using the LHAPDF package --
the group is specified by a character string and the set by an integer.
Please see {\tt http://vircol.fnal.gov/} for further details.

\begin{center}
\{blank line\} \\
{\tt [Jet definition and event cuts] }
\end{center}

\item {\tt m34min}, {\tt m34max}, {\tt m56min}, {\tt m56max}.
These parameters represent a basic set of cuts that may be applied
to the calculated cross-section. The only events that contribute to
the cross-section will have, for example,
{\tt m34min} $<$ {\tt m34} $<$ {\tt m34max} where {\tt m34} is the
invariant mass of particles 3 and 4 that are specified by {\tt nproc}.

\item {\tt inclusive}.  This logical parameter chooses whether the
calculated cross-section should be inclusive in the number of jets
found at NLO. An {\em exclusive}
cross-section contains the same number of jets at next-to-leading
order as at leading order. An {\em inclusive} cross-section may
instead contain an extra jet at NLO.

\item {\tt algorithm} This specifies the jet-finding algorithm that
is used, and can take the values
{\tt ktal} (for the Run II $k_T$-algorithm) and {\tt cone} (for
a midpoint cone algorithm).

\item {\tt ptmin\_jet, etamin\_jet, etamax\_jet}. These specify the values
of $p_T^{\rm min}$, $|\eta|^{\rm min}$ and $|\eta|^{\rm max}$ for the
jets that are found by the algorithm. 

\item {\tt Rcut\_jet}. If the final state of the chosen process contains
either quarks or gluons then for each event an attempt will be made
to form them into jets. For this it is necessary to define the
jet separation $\Delta R=\sqrt{{\Delta \eta}^2 + {\Delta \phi}^2}$
so that after jet combination, all jet pairs are separated by
$\Delta R >$~{\tt Rcut\_jet}.

\item {\tt makecuts}. If this parameter is set to {\tt .false.} then
no additional cuts are applied to the events and the remaining
parameters in this section are ignored. Otherwise, events will
be rejected according to a set of cuts that is specified below.
Further options may be implemented by editing {\tt src/User/gencuts.f}
-- please contact the authors for guidance if necessary.

\item {\tt ptmin\_lepton, ymax\_lepton}. These specify the values
of $p_T^{\rm min}$ and $|y|^{\rm max}$ for the hardest lepton produced
in the process.
\item {\tt ptmin\_missing}. Specifies the minimum missing transverse
momentum (coming from neutrinos).
\item {\tt ptmin\_lepton(2nd+), etamax\_lepton(2nd+)}. These specify
the values of $p_T^{\rm min}$ and $|y|^{\rm max}$ for the remaining
leptons in the process. This allows for staggered cuts where, for
instance, only one lepton is required to be hard and central.

\item {\tt R(jet,lept)\_min}. Using the definition of $\Delta R$ above,
requires that all jet-lepton pairs are separated by
$\Delta R >$~{\tt R(jet,lept)\_min}.

\item {\tt R(lept,lept)\_min}. When non-zero, all lepton-lepton pairs
must be separated by $\Delta R >$~{\tt R(lept,lept)\_min}.

\item {\tt Deltay(jet,jet)\_min}. This enforces a rapidity gap between
the two hardest jets, so that
$|y^{\rm jet~1} - y^{\rm jet~2}| >$~{\tt Deltay(jet,jet)\_min}.

\item {\tt ptmin\_bjet,  etamax\_bjet}. If {\tt makecuts} is {\tt .true.}
and a process involving $b$-quarks is being calculated, then these can
be used to specify {\em stricter} values of $p_T^{\rm min}$
and $|\eta|^{\rm max}$ for $b$-jets.

\item {\tt ptminphoton, etamaxphoton}. These specify the values
of $p_T^{\rm min}$ and $|y|^{\rm max}$ for any photons produced.

\item {\tt conephoton, coneptcut}. These constitute a photon isolation
cut which ensures that the amount of hadronic
transverse momentum in a cone around each photon is less than
a specified fraction of the photon's $p_T$.
\begin{displaymath}
\sum_{R < R_0} p_T^{\rm hadronic} < f \times p_T^{photon},
\end{displaymath}
where $R_0$ and $f$ are specified by {\tt cone\_photon} and
{\tt cone\_ptcut} respectively.

\begin{center}
\{blank line\} \\
{\tt [Anomalous couplings of the W and Z] }
\end{center}

\item {\tt Delta\_g1(Z)}. {\it See section~\ref{subsec:diboson}.}
\item {\tt Delta\_K(Z)}. {\it See section~\ref{subsec:diboson}.}
\item {\tt Delta\_K(gamma)}. {\it See section~\ref{subsec:diboson}.}
\item {\tt Lambda(Z)}. {\it See section~\ref{subsec:diboson}.}
\item {\tt Lambda(gamma)}. {\it See section~\ref{subsec:diboson}.}
\item {\tt Form-factor scale, in TeV}. {\it See section~\ref{subsec:diboson}.}

\begin{center}
\{blank line\} \\
{\tt [How to resume/save a run] }
\end{center}

\item {\tt readin}. If {\tt .true.}, the program will read in a
previously saved {\tt VEGAS} grid from the file specified by
{\tt ingridfile.grid}. Note that this, and the following 3 options,
have no effect if {\tt part} is set to {\tt tota} (in this case, grids
are automatically saved and loaded).

\item {\tt writeout}. If {\tt .true.}, the program will write out
the {\tt VEGAS} grid at the end of the run, to the file specified by
{\tt outgridfile.grid}.

\item {\tt ingridfile}.  {\it See above.}

\item {\tt outgridfile}.  {\it See above.}

\end{itemize}

\section{Output}
 
In addition to the direct output of the program to {\tt stdout}, after
the final sweep of {\tt VEGAS} the program will output two additional files.
If a working directory was specified in the command line, then these
output files will be written to that directory.

The standard output will detail the iteration-by-iteration best estimate
of the total cross-section, together with the accompanying error estimate.
After all sweeps have been completed, a final summary line will be printed.
In the {\tt npart}~$=$~{\tt tota} case, this last line will actually be the
sum of the two separate real and virtual integrations.

The two other output files are {\tt outputname.dat} and
{\tt outputname.top}, which contain data for various histograms associated
with the calculated process. The first of these is in a raw format 
which may be read in by a plotting package of the user's choosing. The
other file contains the histograms as a {\tt TOPDRAWER} file, as well
as a summary of the options file ({\tt input.DAT}) in the form of
comments at the beginning. The structure
of {\tt outputname} is as follows:
\begin{displaymath}
{\tt procname\_part\_pdlabel\_scale\_runstring}
\end{displaymath}
where {\tt procname} is a label assigned by the program corresponding to
the calculated process; the remaining labels are as input by the user
in the file {\tt input.DAT}.

\section{Notes on specific processes}
\label{sec:specific}

\subsection{Di-boson production}
\label{subsec:diboson}

As of version 3.0, it is possible to specify anomalous trilinear
couplings for the $W^+W^-Z$ and $W^+W^-\gamma$ vertices that are
relevant for $WW$ and $WZ$ production. To run in this mode, one
must set {\tt zerowidth} equal to {\tt .true.}, which should be
sufficient anyway, and modify the appropriate lines in {\tt input.DAT}.

The anomalous couplings appear in the Lagrangian,
${\cal L} = {\cal L}_{SM} + {\cal L}_{anom}$ as follows
(where ${\cal L}_{SM}$ represents the usual Standard Model Lagrangian):
\begin{eqnarray}
{\cal L}_{anom} & = & i g_{WWZ} \Biggl[
 \Delta g_1^Z \left( W^*_{\mu\nu}W^\mu Z^\nu - W_{\mu\nu}W^{*\mu} Z^\nu \right)
+\Delta\kappa^Z W^*_\mu W_\nu Z^{\mu\nu} \nonumber \\
 & &+
 \frac{\lambda^Z}{M_W^2} W^*_{\rho\mu} W^\mu_\nu Z^{\nu\rho} \Biggr]
+i g_{WW\gamma} \Biggl[ 
 \Delta\kappa^\gamma W^*_\mu W_\nu \gamma^{\mu\nu}
+\frac{\lambda^\gamma}{M_W^2} W^*_{\rho\mu} W^\mu_\nu\gamma^{\nu\rho}
 \Biggr], \nonumber
\end{eqnarray}
where $X_{\mu\nu} \equiv \partial_\mu X_{\nu} - \partial_\nu X_{\mu}$
and the overall coupling factors are $g_{WWZ}=-e$,
$g_{WW\gamma}=-e\cot\theta_w$.
This is the most general Lagrangian that conserves $C$ and $P$
separately and electromagnetic gauge invariance requires that there
is no equivalent of the $\Delta g_1^Z$ term for the photon coupling.

In order to avoid a violation of unitarity, these couplings are included
in {\tt MCFM} only after suppression by dipole form factors,
\begin{displaymath}
\Delta g_1^Z \rightarrow \frac{\Delta g_1^Z}{(1+\hat{s}/\Lambda^2)^2}, \qquad
\Delta \kappa^{Z/\gamma} \rightarrow
 \frac{\Delta \kappa_1^{Z/\gamma}}{(1+\hat{s}/\Lambda^2)^2}, \qquad
\lambda^{Z/\gamma} \rightarrow
 \frac{\Delta \lambda^{Z/\gamma}}{(1+\hat{s}/\Lambda^2)^2},
\end{displaymath}
where $\hat{s}$ is the vector boson pair invariant mass and $\Lambda$
is an additional parameter giving the scale of new physics, which should
be in the TeV range.
These form factors should be produced by the new physics associated with the
anomalous couplings and this choice is somewhat arbitrary.

The file {\tt input.DAT} contains the values of the $6$ parameters
which specify the anomalous couplings:
\begin{verbatim}
    0.0d0           [Delta_g1(Z)]
    0.0d0           [Delta_K(Z)]
    0.0d0           [Delta_K(gamma)]
    0.0d0           [Lambda(Z)]
    0.0d0           [Lambda(gamma)]
    2.0d0           [Form-factor scale, in TeV]
\end{verbatim}
with the lines representing $\Delta g_1^Z$, $\Delta \kappa^Z$,
$\Delta \kappa^\gamma$, $\lambda^Z$, $\lambda^\gamma$ and
$\Lambda$~[TeV] respectively. By setting the first 5 parameters to zero,
as above, one recovers the Standard Model result.

\subsection{$W/Z+2$~jets production}
\label{subsec:v2j}

For these processes, the next-to-leading order matrix elements are
particularly complex and so they have been divided into two groups.
The division is according to the lowest order diagrams from which they
originate:
\begin{enumerate}
\item Diagrams involving two external quark lines and two external gluons,
the ``{\tt Gflag}'' contribution. The real diagrams in this case thus
involve three external gluons.

\item Diagrams where all four external lines are quarks,
the ``{\tt Qflag}'' contribution. The real diagrams in this case 
involve only one gluon.
\end{enumerate}

By specifying {\tt Gflag} and {\tt Qflag} in {\tt input.DAT} one may
select one of these options at a time. The full result may be obtained
by straightforward addition of the two individual pieces, with no
meaning attached to either piece separately. In the lowest order calculation,
both of these may be set to {\tt .true.} simultaneously - however this is
the only case where this is possible.


\subsection{$H +$~jet production}

The production a standard model Higgs is included in the infinite top
limit mass limit using an effective Lagrangian approach. The production of 
a Higgs + one jet (process 271) is included using the virtual matrix elements 
calculated in refs.~\cite{Ravindran:2002dc} and \cite{Schmidt:1997wr}.
Phenomenological results have previously been 
given in refs.~\cite{deFlorian:1999zd},\cite{Ravindran:2002dc} 
and \cite{Glosser:2002gm}.

\begin{thebibliography}{99}
%
%\cite{Maltoni:2002qb}
\bibitem{Maltoni:2002qb}
F.~Maltoni and T.~Stelzer,
%``MadEvent: Automatic event generation with MadGraph,''
JHEP {\bf 0302}, 027 (2003)
[arXiv:hep-ph/0208156].
%%CITATION = HEP-PH 0208156;%%

%\cite{deFlorian:1999zd}
\bibitem{deFlorian:1999zd}
D.~de Florian, M.~Grazzini and Z.~Kunszt,
%``Higgs production with large transverse momentum in hadronic collisions  at next-to-leading order,''
Phys.\ Rev.\ Lett.\  {\bf 82}, 5209 (1999)
[arXiv:hep-ph/9902483].
%%CITATION = HEP-PH 9902483;%%

\bibitem{Ravindran:2002dc}
V.~Ravindran, J.~Smith and W.~L.~Van Neerven,
%``Next-to-leading order QCD corrections to differential distributions of  Higgs boson production in hadron hadron collisions,''
Nucl.\ Phys.\ B {\bf 634}, 247 (2002)
[arXiv:hep-ph/0201114].
%%CITATION = HEP-PH 0201114;%%

%\cite{Schmidt:1997wr}
\bibitem{Schmidt:1997wr}
C.~R.~Schmidt,
%``H $\to$ g g g (g q anti-q) at two loops in the large-M(t) limit,''
Phys.\ Lett.\ B {\bf 413}, 391 (1997)
[arXiv:hep-ph/9707448].
%%CITATION = HEP-PH 9707448;%%

%\cite{Glosser:2002gm}
\bibitem{Glosser:2002gm}
C.~J.~Glosser and C.~R.~Schmidt,
%``Next-to-leading corrections to the Higgs boson transverse momentum  spectrum in gluon fusion,''
JHEP {\bf 0212}, 016 (2002)
[arXiv:hep-ph/0209248].
%%CITATION = HEP-PH 0209248;%%

\end{thebibliography}


\end{document}
